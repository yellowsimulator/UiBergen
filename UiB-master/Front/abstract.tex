% !TEX encoding = UTF-8 Unicode
%!TEX root = ../Main/thesis.tex
% !TEX spellcheck = en-US
%%=========================================
\documentclass[../Main/thesis.tex]{subfiles}
\begin{document}
\null\vfill
\addcontentsline{toc}{chapter}{Abstract}
\chapter*{Abstract}

The 21st century, characterized by the rise of global economy, relies to a great degree on machines of all sorts. From trains, cars, airplanes for transportation, to heavy machines for goods production such as oil, gas, food, etc. Operating in harsh condition, machines are prone to failure, whose consequences can incur financial losses, the degradation of the environment and in a worst case scenario the loss of human life. 
\justify
 One of the central component of nearly all machines are bearings. They allow rotation movement and are subjected to large loads continuously. Consequently, bearing failure represents more than 40 percent of machine failure. To mitigate bearing degradation, they are monitored, by measuring and recording their vibration through a sensor. A vibration measurement is called a time signal. Bearings degradation in general occurs at a specific frequency called the defect frequency. Therefore the core of bearing monitoring is to detect as early as possible the defect frequency and take appropriate actions.
 \justify
 To detect the defect frequency, the bearing vibration time signal is transformed to a frequency spectrum. However, due to the dynamic of bearings, a direct transformation of the time signal to a frequency spectrum is inefficient in detecting the failure frequency. To circumvent this issue a method called the high frequency resonance technique was introduced. Its transform the time signal through a series of operations in order to extract the relevant signal, before extracting the frequency spectrum though the fast Fourier transform. 

%Two methods for bearings fault detection are presented. Namely: Hilbert Huang transform and wavelet transform coupled with machine learning. Hilbert Huang transform is a data decomposition procedure, that uses local properties of a signal, to iteratively extract its sub-components. It deals with non-linearity and non-stationary with ease by design. It is coupled with a robust seasonal trend decomposition method, to extract high frequency short duration pulses, emitted by a crack located in the outer ring of a bearing. Wavelet transform on the other hand, uses a priori basis functions called wavelets, to decompose and reconstruct signals. It extracts both temporal and frequency information, and is able to \say{zoom in} and \say{zoom out} of data. The former and the latter properties are due to a scaling factor build in wavelet by design.
%We coupled wavelet transform with support vector machine, which is a linear classification algorithm, to monitor and detect bearing defect for a system of four bearings.

\vfill\vfill
\clearpage
\blankpage

\end{document}
