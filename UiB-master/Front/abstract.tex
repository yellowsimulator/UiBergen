% !TEX encoding = UTF-8 Unicode
%!TEX root = ../Main/thesis.tex
% !TEX spellcheck = en-US
%%=========================================
\documentclass[../Main/thesis.tex]{subfiles}
\begin{document}
\null\vfill
\addcontentsline{toc}{chapter}{Abstract}
\chapter*{Abstract}
We present two methods for bearings fault detection. Namely: Hilbert Huang transform and wavelet transform coupled with machine learning. Hilbert Huang transform is a data decomposition procedure, that uses local properties of a signal, to iteratively extract its sub-components. It deals with non-linearity and non-stationary with ease by design. It is coupled with a robust seasonal trend decomposition method, to extract high frequency short duration pulses, emitted by a crack located in the outer ring of a bearing. Wavelet transform on the other hand, uses a priori basis functions called wavelets, to decompose and reconstruct signals. It extracts both temporal and frequency information, and is able to \say{zoom in} and \say{zoom out} of data. The former and the latter properties are due to a scaling factor build in wavelet by design.
We coupled wavelet transform with support vector machine, which is a linear classification algorithm, to monitor and detect bearing defect for a system of four bearings.

\vfill\vfill
\clearpage
\blankpage

\end{document}
