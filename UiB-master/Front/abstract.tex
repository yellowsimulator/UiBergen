% !TEX encoding = UTF-8 Unicode
%!TEX root = ../Main/thesis.tex
% !TEX spellcheck = en-US
%%=========================================
\documentclass[../Main/thesis.tex]{subfiles}
\begin{document}
\null\vfill
\addcontentsline{toc}{chapter}{Abstract}
\chapter*{Abstract}
Since the industrial revolution, machine of all sorts without doubt, have been, and still are the driving force of the world economy.
Operating in harsh conditions, machines are prone to failure, which can incur financial losses, the degradation of the environment, and sometime human casualties. 
\justify
 One of the main component of nearly all machines are bearings. Due to their  geometrical characteristics, the latter facilitate rotation movements while being continuously subjected to large loads . Consequently, bearing failures represent more than 40$\%$ of machine breakdown. To mitigate bearing degradation, they are monitor in order to detect incipient faults as early as required. This is achieved by mounting sensors on strategics area   surrounding a bearing, in order to measure, collect and analyze its vibration movement (vibration signal). 
 \justify
The analysis processes for bearing fault detection in most industrial applications, rely on the Fourier transform. This is due to the fact that bearings faults occur at specific frequencies, called characteristic defect frequencies. A defect in a bearing will emit a periodic high frequency pulse as the bearing rotates. The frequency at which the pulse is emitted is the defect frequency.
 As a result, the core of bearing monitoring, is the early detection of onset of failure, expressed by the defect frequencies. The latter can be found in the frequency spectrum, derived by the Fourier transform.
 \justify
 Bearings are in general mounted on machines comprising several components, each emitting its own vibration signal.
 This render the detection of defect frequencies in a bearing frequency spectrum challenging, due to interfering  signals, and induced noise. Therefore, the frequency spectrum obtained directly from a bearing vibration signal, contains little diagnostic information. To circumvent the latter issue, a series of signal filtering operations and mathematical transformations are applied to a bearing vibration signal. The goal is to isolate and expose the relevant signal, that contains the defects frequencies. This process can be achieved through various methods. However, the most prevalent scheme is the so called high frequency resonance technique (HFRT). The HFRT removes or dump irrelevant signals, and extract relevant one, before applying the Fourier transform. This result in a bearing frequency spectrum, that reveals the defect frequencies.
 \justify
 Although efficient and widely used, the high frequency resonance technique can generate a noisy spectrum, in particular when bearing defects are severe. This can render bearings faults detection challenging. To circumvent the above challenge, this thesis posit a new scheme, that relies on a set of filtering techniques, in order to generate a noiseless spectrum. To test the validity of the proposed new method, it is applied to a case study. The results derived from this test, proves that the new scheme is able to reveal bearing defect frequencies, in a relatively noiseless spectrum, as opposed to the HFRT.


\vfill\vfill
\clearpage
\blankpage

\end{document}
