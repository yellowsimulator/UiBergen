% !TEX encoding = UTF-8 Unicode
%!TEX root = ../Main/thesis.tex
% !TEX spellcheck = en-US
%%=========================================
\documentclass[../Main/thesis.tex]{subfiles}
\begin{document}
\null\vfill
\addcontentsline{toc}{chapter}{Abstract}
\chapter*{Abstract}
Since the industrial revolution, machine of all sorts without doubt, have been, and still are the driving force of the world economy.
Operating in harsh conditions, machines are prone to failure, which can incur financial losses, the degradation of the environment, and sometime human casualties. 
\justify
 One of the main component of nearly all machines are bearings. Due to their  geometrical characteristics, the latter facilitate rotation movements while being continuously subjected to large loads . Consequently, bearing failures represent more than 40$\%$ of machine breakdown. To mitigate bearing degradation, they are monitor in order to detect incipient faults as early as required. This is achieved by mounting sensors on strategics area   surrounding a bearing, in order to measure, collect and analyze its vibration movement (vibration signal). 
 \justify
The analysis process for bearing fault detection in most industrial applications, rely on the Fourier transform, which convert a bearing vibration signal to its corresponding frequency spectrum. This is due to the fact that bearings faults occur at specific frequencies, called characteristic defect frequencies. A defect in a bearing will emit a periodic high frequency pulse as the bearing rotates. The frequency at which the pulse is emitted is the defect frequency.
 As a result, the core of bearing monitoring, is the early detection of onset of failure, expressed by the defect frequencies, that can be found in the frequency spectrum, derived by the Fourier transform.
 \justify
 Bearings are in general mounted on machines comprising several components, each emitting its own vibration signal.
 This render the detection of defect frequencies in a bearing frequency spectrum challenging, due to interfering  signals, inducing noise. Therefore, the frequency spectrum obtained directly from a bearing vibration signal contains little diagnostic information. To circumvent the latter issue, a series of signal filtering operations and mathematical transformations are applied to a bearing vibration signal, in order to isolate and expose the relevant signal, that contains the defects frequencies. This process can be achieved though the so called high frequency resonance technique (HFRT) method. The latter remove or dump irrelevant signals before applying the Fourier transform in order to generate a \say{clean} bearing frequency spectrum, that reveals the defect frequencies.
% \justify
% Although efficient in bearing fault detection, the HFRT relies solely on the Fourier transform to obtain the frequency spectrum. Since the Fourier transform operates in the frequency domain, it can be challenging to detect temporal events such as pulses emitted by bearing faults. Mathematically, those are signals with compact support. In addition, the Fourier transform uses predefined trigonometric basis functions in decomposing a signal to its frequency spectrum. As a result, if a signal is frequency or amplitude modulated, the frequency spectrum obtained can be noisy, and detecting bearing failure frequency accurately becomes challenging. To circumvent this problem, this thesis introduces the Hilbert Huang transform (HHT) for bearing fault detection. As opposed to the Fourier transform, the HHT operates in the time domain. It is not bound by predefined basis functions such as trigonometric extensions in Fourier transform. This property allows the Hilbert Huang transform to adaptively decomposed a signal in the time domain, in terms of nearly mono component signals. Here adaptively means there is no basis functions for the decomposition. This gives the Hilbert Huang transform a leverage over the Fourier transform, as it is able to reveal salient properties of data in the time domain, such as pulses emitted by bearing defects.
% \justify
% In this thesis, by applying the Hilbert Huang transform to bearing fault detection, a completely noise less frequency spectrum is extracted from the bearing vibration signal, and the bearings failure frequencies are clearly identifiable. With the present method, pulses emitted by bearing failure can be visualized, allowing a better analysis in bearing fault detection.
%Two methods for bearings fault detection are presented. Namely: Hilbert Huang transform and wavelet transform coupled with machine learning. Hilbert Huang transform is a data decomposition procedure, that uses local properties of a signal, to iteratively extract its sub-components. It deals with non-linearity and non-stationary with ease by design. It is coupled with a robust seasonal trend decomposition method, to extract high frequency short duration pulses, emitted by a crack located in the outer ring of a bearing. Wavelet transform on the other hand, uses a priori basis functions called wavelets, to decompose and reconstruct signals. It extracts both temporal and frequency information, and is able to \say{zoom in} and \say{zoom out} of data. The former and the latter properties are due to a scaling factor build in wavelet by design.
%We coupled wavelet transform with support vector machine, which is a linear classification algorithm, to monitor and detect bearing defect for a system of four bearings.

\vfill\vfill
\clearpage
\blankpage

\end{document}
