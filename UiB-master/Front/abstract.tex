% !TEX encoding = UTF-8 Unicode
%!TEX root = ../Main/thesis.tex
% !TEX spellcheck = en-US
%%=========================================
\documentclass[../Main/thesis.tex]{subfiles}
\begin{document}
\null\vfill
\addcontentsline{toc}{chapter}{Abstract}
\chapter*{Abstract}
Machine of all sorts have been and still are the driving force of global economy since the industrial revolution.
Operating in harsh conditions, machines are prone to failure which can incur financial losses, the degradation of the environment and in worst cases the loss of human life. 
\justify
 One of the central component of nearly all machines are bearings. They facilitate rotation movements and are subjected to large loads continuously. Consequently, bearing failures represent more than 40 percent of machine breakdown. To mitigate
 bearing degradation, a set of sensors mounted on strategics area surrounding the bearings, collect vibration signals periodically. Each vibration signal is converted to its frequency spectrum through the Fourier transform. The latter is a corner stone in bearing fault detection. This is due to the fact that bearing faults occur at specific frequencies, called characteristic defect frequencies, which are derived from the bearing geometrical characteristics. A defect in a bearing will emit a periodic high frequency pulse as the bearing rotates. The frequency at which the pulse is emitted is the characteristic defect frequencies.
 The core of bearing monitoring is the early detection of onset of failure in order to take appropriate actions.
 \justify
 Bearings are mounted on a motor which is comprise of several components. Therefore bearings vibration signals coexist with other machine component vibration signals. Consequently a direct transformation of a bearing vibration signal to a frequency spectrum is inefficient in detecting bearing failure frequencies. To circumvent the latter issue, a method called the high frequency resonance technique (HFRT) was introduced. Its transform the vibration signal through a series of filtering and mathematical operations, in order to extract the relevant signal, before extracting the frequency spectrum though the fast Fourier transform. 
 \justify
 Although efficient in bearing fault detection, the HFRT relies solely on the Fourier transform to obtain the frequency spectrum. Since the Fourier transform operates on the frequency domain, it ignores local temporal event. As a consequence, the frequency spectrum obtained is often noisy. To circumvent this problem, This thesis introduce the Hilbert Huang transform (HHT) for bearing fault detection. As opposed to the Fourier transform, the HHT operates in the time domain.
 By applying the HHT to bearing fault detection, a completely noise less frequency spectrum is extracted from the bearing vibration signal, and the bearings failure frequencies are clearly identifiable. With the present method, pulses emitted by bearing failure can be visualized, allowing a better analysis in bearing fault detection.
%Two methods for bearings fault detection are presented. Namely: Hilbert Huang transform and wavelet transform coupled with machine learning. Hilbert Huang transform is a data decomposition procedure, that uses local properties of a signal, to iteratively extract its sub-components. It deals with non-linearity and non-stationary with ease by design. It is coupled with a robust seasonal trend decomposition method, to extract high frequency short duration pulses, emitted by a crack located in the outer ring of a bearing. Wavelet transform on the other hand, uses a priori basis functions called wavelets, to decompose and reconstruct signals. It extracts both temporal and frequency information, and is able to \say{zoom in} and \say{zoom out} of data. The former and the latter properties are due to a scaling factor build in wavelet by design.
%We coupled wavelet transform with support vector machine, which is a linear classification algorithm, to monitor and detect bearing defect for a system of four bearings.

\vfill\vfill
\clearpage
\blankpage

\end{document}
