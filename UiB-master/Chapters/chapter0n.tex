% !TEX encoding = UTF-8 Unicode
%!TEX root = thesis.tex
% !TEX spellcheck = en-US
%%=========================================
\documentclass[thesis.tex]{subfiles}
\begin{document}
\chapter[Conclusions]{Conclusions, Discussion, and Recommendations for Further Work}
\label{ch:conclusions}
%%This is the last chapter
In this final chapter you should sum up what you have done and which results you have got. 
You should also discuss your findings, and give recommendations for further work.

%%=========================================
\section{Summary and Conclusions}
\label{sec:summary_and_conclusions}
%Here, you present a brief summary of your work and list the main results you have got. 
%You should give comments to each of the objectives in Chapter 1 and state whether or not you have met the objective. 
%If you have not met the objective, you should explain why (e.g., data not available, too difficult).
%
%This section is similar to the Summary and Conclusions in the beginning of your report, but more detailed---referring to the the various sections in the report.
The industrial revolution led to efficiency in all economical sectors, and consequently an improvement in nearly all aspects of human live. All this progress would not have been possible without machines. However, the latter are prone to failure, and must be monitor and maintain, to avoid unexpected and catastrophic breakdown. To mitigate such events, most machines are equipped with an array of sensors, collecting continuously data, which are analyzed in order to detect as early as possible any sign of failure.
\justify
In this thesis, we looked at different methods used to detect bearings defects, which represents more than 40 $\%$ of all deterioration in rotating machines. By allowing rotational movement, bearings are critical part of nearly all rotating equipment. They are continually subjected to extensive load, thereby must be closely kept under observation. The most widely used method for bearing fault detection is the Fourier transform. It decomposes vibration signals into a spectrum of frequencies. For a bearing, the failure frequency is derived from its geometrical components and the rotational speed of the machine on which it is mounted. Once the failure frequency is known, it suffices to search for it in the frequency spectrum, derived from the Fourier transform of the vibration signal.
\clearpage
\justify
In Chapter 2, we showed how the Fourier transform method can be used to decompose a signal into its corresponding frequency spectrum. We decomposed 984 vibration samples each measured from four bearings. By applying the Fourier transform method, we obtained the failure frequencies with corresponding amplitude for each samples. The sample where recorded every 10 minutes. We then plotted the amplitudes of failure frequency as a function of time. From this method it was clear that bearing number four was severely affected by a crack in the outer ring. 

\justify
However, dependency of failure frequency on the rotational speed of the motor, can incur difficulties when the rotational speed varies continuously. In addition, the Fourier transform method, is not well equipped to dealing with non-linear and non-stationary signals. The assumption of an a-priory trigonometrical basis, which are globally define, does not allow for capturing local phenomenon such a pulses emitted by a crack in bearings. 
%%=========================================
\section{Discussion}
\label{sec:discussion}
Here, you may discuss your findings based on your results, their strengths and limitations. 
Note that this discussion is more high level than discussions made in relation to results you have achieved and presented in the previous chapter. 
The discussion here should put your work in larger context. 
You may address if you achieved what you had intended to do, why not (if you did not), if you got results in which you did not expect, why the results are important, why there are limitations in using the results, or if there are opportunities to transfer your results and findings into other domains, and so on.
%%=========================================
\section{Recommendations for Further Work}
\label{sec:recommendations_for_further_work}
You should give recommendations to possible extensions to your work. 
The recommendations should be as specific as possible, preferably with an objective and an indication of a possible approach.

The recommendations may be classified as:
\begin{itemize}
\item Short-term
\item Medium-term
\item Long-term
\end{itemize}

\end{document}
