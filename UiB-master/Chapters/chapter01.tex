% !TEX encoding = UTF-8 Unicode
%!TEX root = ../Main/thesis.tex
% !TEX spellcheck = en-US
%%=========================================
\documentclass[../Main/thesis.tex]{subfiles}
\begin{document}
\chapter{Introduction}
\label{ch:introduction}
%The first chapter of a well-structured thesis is always an introduction, setting the scene with background, problem description, objectives, limitations, and then looking ahead to summarize what is in the rest of the report. 
%This is the part that readers look at first---\emph{so make sure it hooks them!}

%%=========================================
\section{Background}
\label{sec:background}
%In this section, you should present the problem that you are going to investigate or analyze; why this problem is of interest; what has, so far, been done to solve the problem, and which parts of the problem that remain.
%{\color{red}Below, I have set up some headings (subsection titles) without a number. 
%These are included to help you remember to cover the related issues. 
%The headings should be removed in your final print.}
%%=========================================
%\subsection*{Problem Formulation}
Since the industrial revolution, engines and machines have been the driving force for economical growth across industries such as automotive, airline, oil and gaze, to name a few. However, machines are prone to failure and must be monitor and maintain regularly to avoid catastrophic failure which can lead to significant financial and human lost. To mitigate equipment and machines proclivity toward failure and the associated cost, a process called predictive maintenance has been developed within the industrial community. Predictive maintenance  for machines and industrial equipments can be defined as a maintenance philosophy or more generally a framework with a set of methods used to predict and prevent machine failure in order to avoid unexpected downtime and reduce related human and financial cost. This maintenance philosophy, when correctly implemented, increase machine life time and reduce maintenance cost by detecting as early as possible early sign of failure and plan maintenance accordentely.
\justify
In rotating machines, more than $40 \% $ of failure can be attributed to bearing faults, \cite{patidar13}. The most commonly used method in the industry to detect specific bearing faults is the Fast Fourier Transform (FFT). The FFT method consists of transforming a vibration signal emanating from a bearing, to its corresponding frequency spectrum. If the bearing exhibits signs of failure, the frequency of the latter will be visible in the frequency spectrum. It is worth noting that, a bearing failure frequency is derived from the bearing geometry, and depends on the rotating speed of the machine.
\justify
One of the drawback of the FFT method, among others, is the dependence of the failure frequency on the rotating speed of the machine. For constant rotating speed it is straight forward to estimate the failure frequencies. However, when the rotating speed varies continuously, it becomes very challenging to estimate the failure frequencies accurately, and this can lead to incorrect fault diagnosis.
\justify
To circumvent the latter, many efforts have been directed toward alternative methods for bearing fault detection. In this thesis we focused on methods derived from machine learning, Wavelet transform and Hilbert Huang transform. The attractiveness of these methods relies on the fact that, they do not require the bearing geometrical properties and the rotating speed of the machine to detect failure. 
\justify
Wavelet transform is a signal analysis tool such as Fourier transform. In contrast to the latter, wavelets can be used to detect short duration hight frequency burst, \cite{albert09}.
Wavelets are family of functions generated by two functions: the so called mother wavelet and a scaling function. The scaling property allows a wavelet to zoom in on short and high frequency burst, \cite{albert09},
such as tiny onset of cracks in bearings.
\justify
Hilbert Huang transform was developed recently by  Huang, \cite{huang98} to deal efficiently with non linear and non stationary processes. It decomposes data adaptively 
into its elementary parts by using an a-priory basis functions. Unlike in Fourier transform where trigonometric functions are used to decomposed data, adaptive decomposition means that 
the basis functions are completely determined by the data itself, \cite{huang08}. This allows in theory, to access intrinsic properties of data.
\justify
Machine learning methods commonly used are supervised and unsupervised learning methods. In supervised learning, an algorithm learns the pattern of instances of failure, in order to detect them in subsequent samples. It is however common that failure instances are not available or are too scarce to learn from. In that event, unsupervised learning algorithms are the alternative to supervised learning. The former algorithms, categorize data into different groups where data points in the same cluster share some sort of similarity.
\justify
The use of signal analysis tool such as wavelet and Hilbert Huang transform, as well as machine learning for bearing failure have been studied by many researchers. To outline the contribution of this thesis, we present a literature survey in section \ref{sec:relatedwork}, followed by our contribution in section \ref{sec:contributions}. In chapter \ref{sec:chapter2} we show how Fourier analysis can be applied to bearing fault detection by using a concrete case study. In Chapter \ref{sec:chapter3} we show how wavelet transform and a robust statistical dispersion measure can be used to estimate the health of a bearing.
In chapter \ref{sec:ml}, we use wavelet transform and machine learning to classify bearing failure. Chapter \ref{sec:hht} presents Hilbert Huang transform and how it is used for bearing failure frequency estimation. The thesis is concluded in \ref{sec:conclusion} where we also show the limitation of our contribution and present future work. That being said, let rock!.
%The detailed treatment of these methods and the 
%%=========================================
\section{Related work}
\label{sec:relatedwork}
%You should here present the main books and articles that treat problems that are similar to what  you are studying, and give proper references to each of these as they are reported. 
%If you,  later in your thesis, describe the ``state of the art'' -- with a detailed literature survey, you may just give a very brief survey here (approx. a quarter of a page). 
%If this is the only literature survey, you need to go into more details. 
%An objective of the literature survey is to show the reader that you are familiar with the main literature within your field of research -- so that you do not ``reinvent the wheel.''


%%=========================================
%\subsection*{What Remains to be Done?}
%After you have defined and delimited your problem -- and presented the relevant results found in the literature within this field, you should sum up which parts of the problem that remain to be solved.
%%=========================================

\section{Contributions }
\label{sec:contributions}
The contribution of this work can be summed up in three. Firstly, We show that a bearing health can be estimated by a robust statistical dispersion measure: the inter quantile range (IQR). We call the latter the health index. Furthermore,  the health index is used to estimate a bearing health trajectory, based on a robust, and computational efficient trend estimation method. The health trajectory is the different stages a bearing can go through until failure occurs.
\justify
Secondly, we estimated a health decision boundary for a bearing, based on a wavelet decomposition, the IQR, and a one class support vector machine. A health decision boundary allows for deciding if the bearing is about to fail or not. For each successive vibration signal, a discrete wavelet transform is used to decompose the vibration signal into two new features: The high and low frequency component. The health index of the high and low frequency components are computed. In this fashion, we obtain a set of points in the high and low frequency feature space. By training a support vector machine we can estimate a health decision boundary for a bearing.
\justify
Lastly, by using the Hilbert Huang transform, and a robust seasonal estimation method, we estimated the failure signal. The failure signal is the signal emitted when a bearing passes a crack. it is a short and high frequency signal indicating the presence for example of a crack in the bearing.
%%=========================================

\blankpage
\end{document}
