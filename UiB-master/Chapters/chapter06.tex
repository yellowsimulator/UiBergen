% !TEX encoding = UTF-8 Unicode
%!TEX root = thesis.tex
% !TEX spellcheck = en-US
%%=========================================
\documentclass[thesis.tex]{subfiles}
\begin{document}
	\chapter[Conclusions]{Conclusions, Discussion, and Recommendations for Further Work}
	\label{ch:conclusions}
	%%This is the last chapter
	%In this final chapter you should sum up what you have done and which results you have got. 
	%You should also discuss your findings, and give recommendations for further work.
	
	%%=========================================
	\section{Summary and Conclusions}
	\label{sec:summary_and_conclusions}
	%Here, you present a brief summary of your work and list the main results you have got. 
	%You should give comments to each of the objectives in Chapter 1 and state whether or not you have met the objective. 
	%If you have not met the objective, you should explain why (e.g., data not available, too difficult).
	%
	%This section is similar to the Summary and Conclusions in the beginning of your report, but more detailed---referring to the the various sections in the report.
	The industrial revolution led to efficiency in all economical sectors, and consequently an improvement in nearly all aspects of human live. All this progress would not have been possible without machines. However, they are prone to failure, and must be monitor and maintain, to avoid unexpected and catastrophic breakdown. To mitigate such events, most machines are equipped with an array of sensors, collecting continuously data. Several methods and algorithms make it possible to analyzed the data generated, in order to detect as early as possible any sign of failure.
	\justify
	In this thesis, we derived different methods applied to bearings defects detection, which represent more than 40 $\%$ of all deterioration in rotating machines. By allowing rotational movement, bearings are critical part of nearly all rotating equipment. They are continually subjected to extensive load, thereby must be closely kept under observation. The most widely used method for bearing fault detection is the Fourier transform based method. It filters and decomposes vibration signals into a spectrum of frequencies. For a bearing, the failure frequency is derived from its geometrical components and the rotational speed of the machine on which it is mounted. Once the failure frequency is known, it suffices to search for it in the frequency spectrum, derived from the Fourier transform of a bearing vibration signal. To set a reference method, the Fourier transformed based method was introduced, followed by two new methods that extend the limitations of the latter.
	\justify
	In Chapter 2, we introduced the Fourier transform based method (FT), which is widely applied in many industry for bearings faults detection. We showed how the Fourier transform method can be used to decompose a signal into its corresponding frequency spectrum, and detect a failure in the outer race of a bearing. We decomposed a series of vibration measurement, sampled in time. By applying the Fourier transform method, we where able to identify a failed bearing by monitoring the failure frequency of four bearings mounted on a motor.
	\justify
	However, when using the Fourier transform method, the dependency of failure frequency on the rotational speed of the motor, can incur difficulties for continuously varying rotational speed. In addition, the Fourier transform method is not well equipped in dealing with non-linear and non-stationary signals. The assumption of an a-priory trigonometrical basis, which are globally define, limits the FT for capturing local phenomenon such high frequency pulses emitted by a crack in bearings.
	\justify
	To circumvent some of the short coming of the Fourier transform, the Hilbert Huang transform (HHT) was introduced in Chapter 3.
	This method adaptively decomposes a signal into sub-signals called intrinsic mode functions (IMF), by a process called empirical mode decomposition (EMD). It uses local properties and statistical estimator such as the mean to iteratively
	\say{siffer} a signal into its sub-components. This allows decomposing non-stationary and nonlinear data with ease.
	By applying the EMD coupled with a robust seasonal trend decomposition method, it was possible to extract high frequency, short duration pulses emitted by a crack located in the outer ring of a bearing.
	\justify
	In chapter 4, we introduced a method that uses the duo wavelet transform-support vector machine, to monitor and detect outer race defect for a system of four bearings attached to a motor. The new methodology consists of three stages: In the first stage, the wavelet transform decomposes a bearing vibration signal into two additional signals. They were called frequency and temporal features. The former and the latter encode frequency and temporal information, and together form the temporal-frequency feature space. In the second stage, the interquartile range (IQR) of the frequency and temporal feature are estimated, for successive vibration samples measured in time. The IQR which is a measure of statistical dispersion, quantifies the health of a sample. A higher IQR implies an \say{unhealthy} sample. In doing so, we obtained several points in the temporal-frequency feature space.
	\justify
	In the last stage, a one class support vector machine (SVM) classifier was applied to the set of points generated in the temporal-frequency feature space. The SVM, which is a linear classification algorithm, was able to set a boundary between \say{healthy} and \say{unhealthy} samples.
	This new methodology was able to detect earlier the onset of failure for one of the bearings which was severely affected by a defect in the outer race.
	
	
	
	%%=========================================
	\section{Discussion}
	\label{sec:discussion}
	The contribution of this thesis rests on two methods: Hilbert Huang transform coupled with the seasonal trend base on Loess (HHT-STL) and wavelet transform coupled with support vector machine (WT-SVM). Both methods are simple and require less transformation as opposed to the Fourier transform based method (band pass filter, high pass filter, low pass filter, Fast Fourier transform). They only require data and work in the time domain, where is the Fourier transform based method also requires the knowledge of the bearing geometrical properties and the rotational speed of the motor.
	\justify
	The HHT-STL was only used to extract high frequency short duration pulses. Once such pulses are extracted, their frequency and amplitude represent the failure characteristics of a bearing. To monitor a bearing we just need to track the amplitude of a pulse. However, at this stage, for this method to be reliable two questions need to be answered 
	\justify
	How to choose he correct intrinsic mode function ?. Does the amplitude and the frequency of the pulse corresponds to the failure frequency on a bearing? 
	The first question is related to the fact that the Hilbert Huang transform generates many intrinsic mode functions. In order, they corresponds to the highest frequency sub-signal to the lower frequency sub-signal
%	Here, you may discuss your findings based on your results, their strengths and limitations. 
%	Note that this discussion is more high level than discussions made in relation to results you have achieved and presented in the previous chapter. 
%	The discussion here should put your work in larger context. 
%	You may address if you achieved what you had intended to do, why not (if you did not), if you got results in which you did not expect, why the results are important, why there are limitations in using the results, or if there are opportunities to transfer your results and findings into other domains, and so on.
	%%=========================================
	\section{Recommendations for Further Work}
	\label{sec:recommendations_for_further_work}
	You should give recommendations to possible extensions to your work. 
	The recommendations should be as specific as possible, preferably with an objective and an indication of a possible approach.
	
	The recommendations may be classified as:
	\begin{itemize}
		\item Short-term
		\item Medium-term
		\item Long-term
	\end{itemize}
	
\end{document}
