% !TEX encoding = UTF-8 Unicode
%!TEX root = thesis.tex
% !TEX spellcheck = en-US
%%=========================================
\documentclass[../Main/thesis.tex]{subfiles}
\begin{document}
	\chapter[Conclusions]{Conclusion}
	\label{sec:conclusions}
	 This chapter sums up the ideas contain in the previous chapters and paragraphs. At first, a succinct summary is put forth, followed by a discussion of the main results, and a list of recommendations to possibly extend this thesis.
	%%This is the last chapter
	%In this final chapter you should sum up what you have done and which results you have got. 
	%You should also discuss your findings, and give recommendations for further work.
	
	%%=========================================
	\section{Conclusions}
	\label{sec:summary_and_conclusions}
	%Here, you present a brief summary of your work and list the main results you have got. 
	%You should give comments to each of the objectives in Chapter 1 and state whether or not you have met the objective. 
	%If you have not met the objective, you should explain why (e.g., data not available, too difficult).
	%
	%This section is similar to the Summary and Conclusions in the beginning of your report, but more detailed---referring to the the various sections in the report.
	The industrial revolution had set up a chain of events, that forged efficiency in all economical sectors, and consequently led to improvements, in nearly all physical aspects of human lives. This progress would not have been possible without machines. However, they are prone to failure, and must be monitor and maintain, to avoid unexpected and catastrophic breakdown. To mitigate such events, most machines are equipped with an array of sensors, collecting continuously data. A myriad of methods and algorithms make it possible to analyze the data generated, in order to detect as early as possible any sign of failure, and take necessary actions.
	\justify
	By facilitating rotational movements, bearings are critical part of nearly all rotating equipment. Being continually subjected to extensive load, bearing failures represent more than 40$\%$ of defect in rotating machines. In most industries, Fourier transform is the pillar of bearing fault detection and is a central part of nearly all methods.
	One of the most widely used scheme is the so called high frequency resonance technique. 
	In the latter, a bearing vibration signal is filtered to remove noise and isolate desirable components, and decomposed into a spectrum of frequencies. For a bearing, the failure frequency is derived from its geometrical components and the rotational speed of the machine on which it is mounted. Once the failure frequencies are known, it suffices to search for them in the frequency spectrum, derived from the Fourier transform of a bearing vibration signal. 
	\justify
	The high frequency resonance technique can produce a very noisy spectrum, in particular when the bearing is subjected to a sever inner race defect. This can render fault detection very challenging. In this thesis therefore, a new method is presented to mitigate this issue.
	This new scheme consists of decomposing an input bearing vibration signal into successive high to low frequency components called intrinsic mode functions (IMFs), through the so called empirical mode decomposition (EMD). The latter is part of the Hilbert-Huang transform, which couples the EMD with the Hilbert transform, in order to derive well define instantaneous frequencies and amplitude. The former and the latter are very important in frequency and amplitude modulated problems.
	\justify
	Once the intrinsic mode functions (IMFs) have been computed, the first IMF is selected and its seasonal trend computed by means of the seasonal trend decomposition method by LOESS (STL). The latter is a computational efficient method to decomposition a signal into its trend and seasonal parts. The former is a monotone function, while the latter is a periodic oscillatory sinusoidal. Furthermore, the power spectrum density of the seasonal component of the first intrinsic mode function is approximated by the periodogram method. This result in a conspicuous energy distribution per frequency contribution, where bearing failure frequencies are clearly identifiable. In addition, bearing failure characteristics such harmonics, and the rotation speed of the motor housing the bearings, are all visible in the estimated spectrum.
	\justify
	The Hilbert-Huang transform, through the empirical mode decomposition (EMD), boasted in generated mono components intrinsic mode functions. However, the EMD suffers from what is known as mode mixing, which makes an IMF not exactly mono component. This is precisely the reason why in this thesis, the seasonal component of each IMF is extracted in order to circumvent the mode mixing issue.
	\justify
	The result obtained are satisfactory in the sens that, bearing failure frequencies are clearly identified. However, only two types of bearing failure were covered in this thesis: namely fault occurring in the inner and outer ring. The justification is that these are the prevalent faults encountered. Therefore the new scheme described in this thesis, could be tested on all types of bearing faults.

	
\end{document}
