% !TEX encoding = UTF-8 Unicode
%!TEX root = ../Main/thesis.tex
% !TEX spellcheck = en-US
%%=========================================
\documentclass[../Main/thesis.tex]{subfiles}
\begin{document}
	\chapter[Conclusions]{Conclusion}
	\label{sec:conclusions}
	 This chapter sums up the ideas put forwards in the previous chapters and paragraphs. In section \ref{sec:comp}, the results obtained from the high frequency technique and the one presented in this thesis are compared.
	 Section \ref{sec:summary_and_conclusions} concludes this thesis with a general summary, followed by a discussion regarding the main results, and recommendations to possibly extend this work.

	
	%%=========================================
	\section{Comparison of results}
	Figure \ref{fig:hfrt-method} shows a frequency spectrum obtained from the high frequency resonance technique, while \ref{fig:yapi-method} shows a frequency spectrum obtained from the scheme proposed in this thesis.
	\begin{figure}[H]
		\centering
		%\includegraphics[width=0.7\linewidth]{../fig/bpfi/first_day_spectrum}
		\includegraphics[width=0.7\linewidth]{../fig/hfrt}
		\caption{Frequency spectrum with an identified ball pass inner race defect frequency obtained from the high frequency resonance technique (HFRT).}
		\label{fig:hfrt-method}
		
		\begin{figure}[H]
			\centering
			%\includegraphics[width=0.8\linewidth]{../fig/periodogram_bpfi/start_imf1_bpfi}
			\includegraphics[width=0.7\linewidth]{../fig/periodogram_bpfi/end_imf1_bpfi}
			\caption{Frequency spectrum with an identified ball pass inner race defect frequency obtained from the method posited in this thesis.}
			\label{fig:yapi-method}
		\end{figure}
	\end{figure}
	\label{sec:comp}
	\section{Conclusions}
	\label{sec:summary_and_conclusions}
	The industrial revolution had set up a chain of events, that forged efficiency in all economical sectors, and consequently led to improvements, in nearly all physical aspects of human lives. This progress would not have been possible without machines. However, their proclivity towards failure, imposes a monitoring and maintenance scheme, in order to avoid unexpected and catastrophic breakdown. To mitigate such events, most machines are equipped with an array of sensors, collecting continuously data. A myriad of methods and algorithms make it possible to analyze the data generated, in order to detect as early as possible any sign of failure, and take necessary actions.
	\justify
	By facilitating rotational movements, bearings are critical part of nearly all rotating equipment. Being continually subjected to extensive load, bearing failures represent more than 40$\%$ of defect in rotating machines. In most industries, Fourier transform is the pillar of bearing fault detection, and is a corner stone of nearly all methods.
	One of the most widely used scheme is the so called high frequency resonance technique. 
	In the latter, a bearing vibration signal is filtered to remove noise, and isolate desirable components, before subsequently being decomposed into a spectrum of frequencies. For a bearing, the failure frequency is derived from its geometrical components and the rotational speed of the machine on which it is mounted. Once the failure frequencies are known, it suffices to search for them in the frequency spectrum, derived from the Fourier transform of a bearing vibration signal. 
	\justify
	The high frequency resonance technique although widely used, can produce a very noisy spectrum, in particular when the bearing is subjected to a sever inner race defect. This can render fault detection very challenging. In this thesis therefore, a new method is presented to mitigate this issue.
	This new scheme consists of decomposing an input bearing vibration signal into successive high to low frequency components called intrinsic mode functions (IMFs), through the so called empirical mode decomposition (EMD). The latter is part of the Hilbert-Huang transform, which couples the EMD with the Hilbert transform, in order to derive well define instantaneous frequencies and amplitude. The former and the latter are very important in frequency and amplitude modulated problems.
	\justify
	Once the intrinsic mode functions (IMFs) have been computed, the first IMF is selected and its seasonal trend computed by means of the seasonal trend decomposition method by LOESS (STL). The latter is a computational efficient method in decomposing a signal into its trend and seasonal parts. The former is a monotone function, while the latter is a periodic oscillatory sinusoidal. Furthermore, the power spectrum density of the seasonal component of the first intrinsic mode function is approximated by the periodogram method(approximation of the power spectral density).
	\justify
	To test the validity of the proposed new method, a case study was performed with bearings vibration data, obtained from an experiment run by the diagnosis branch of the national aeronautical administration (NASA). The application of the new scheme resulted in a conspicuous energy distribution per frequency contribution, where bearing failure frequencies are clearly identifiable. In addition, bearing failure characteristics such harmonics, and the rotation speed of the motor housing the bearings, are all visible in the estimated spectrum.
	\justify
	The Hilbert-Huang transform, through the empirical mode decomposition (EMD), boasts itself in generated mono components intrinsic mode functions. However, the EMD suffers from what is known as mode mixing, which makes an IMF not exactly mono component. This is precisely the reason why in this thesis, the seasonal component of each IMF is extracted in order to circumvent the mode mixing issue.
	\justify
	The result obtained are satisfactory in the sens that, bearing failure frequencies are clearly identified. However, only two types of bearing failure were tested in this thesis: namely fault occurring in the inner and outer ring. The justification is that these are the prevalent faults encountered. As an extension to this work, the new method could be applied to detecting the whole range of bearing failure, in order to further corroborate its full validity. In addition, since the data for this case study was obtained under a control experiment, the new scheme could be applied to data from real industrial processes to broadly justify its validity.

	
\end{document}
