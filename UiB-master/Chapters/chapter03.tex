% !TEX encoding = UTF-8 Unicode
%!TEX root = ../Main/thesis.tex
% !TEX spellcheck = en-US
%%=========================================
\documentclass[../Main/thesis.tex]{subfiles}
\begin{document}
\chapter[Hilbert Huang transform for high frequency pulse detection]{Hilbert Huang transform for high frequency pulse detection}
\label{sec:hht}

Most signal analysis methods such as Fourier transform, impose a-priory basis functions on the signal to be analyzed. In the case of Fourier analysis, the basis functions are trigonometric extensions. Although this implies a rigorous mathematical treatment, the resulting signal decomposition is limited by the mathematical assumptions (\cite{huang98}, \cite{huang08}). Two such assumptions are linearity and stationarity. As most phenomena in nature are non linear and non stationary, this mathematical approach, although rigorous, lacks an important property: Adaptivity (\cite{huang98}, \cite{huang08}). The latter refers to capturing the intrinsic properties of a signal, without imposing an a-priory basis function, (\cite{huang98}, \cite{huang08}). 
\justify 
The Hilbert Huang transform (HHT) was precisely developed to deal with non linear and non stationary processes, in an adaptive fashion (\cite{huang98}). It combines Hilbert spectral analysis with the so call empirical mode decomposition (EMD), to adaptively decompose a signal into its fundamental components called intrinsic mode functions (IMF), (\cite{huang98}).
The richness of the HHT spans from the analysis of differential equations, to the study of geophysical phenomena, as well as bearings faults detection (\cite{huang08},\cite{li2009}, \cite{yan2006} ,\cite{soualhi2015}, \cite{sallo2019}), and by no mean limited to them.
\justify
 The HHT has been successful applied to the analysis of solutions of nonlinear differential equations, such as the Duffing and the Lorentz equations. The intrinsic frequency, the forcing function and the low intensity subharmonics of the numerical solutions for the nonlinear Duffing equation has been extracted through the EMD process, (\cite{huang98}).
 \justify
  The decomposition of the solutions of Lorentz equation, revealed \say{transient components with different frequencies and damping characteristics}, which agreed with previous studies, (\cite{huang98}). The HHT application to seismic waves propagation, made it possible to identified high and low frequency seismic waves (\cite{vasudevan2000}). In particular, its decomposition of the seismic waves induced by the 1999 Taiwan earthquake, revealed that \say{the Fourier transform underestimated low frequency energy} (\cite{huang2001}). 
\justify
The Hilbert Huang transform emerged as a general signal decomposition tool, and in theory can be applied to any signals.
This chapter presents a new method for bearing fault detection. The method couples the HHT with a robust seasonal trend decomposition method called STL to detect bearing faults. STL stands for seasonal trend decomposition based on LOESS.  Section \ref{sec:emd} presents the empirical mode decomposition (EMD), which is the back bone in decomposing a signal adaptively.
\justify
 Section \ref{sec:pulse}, through a concrete case study, shows that the HHT, coupled with the STL method, are able to recover the high frequency pulse like signals, emitted by bearing defects. The section starts by recalling the characteristics of two bearing faults: The outer race defect and the inner race defect. Furthermore, the experiments that generated the data for this case study are described. Afterwards, the seasonal trend decomposition based on LOESS (STL) is presented, followed by a description of the proposed method for bearing fault detection. Finally, the results of the application to the case study are presented. 
\justify
 %As emphasized in the greeks mythology, any hero has an Achilles hill, and the HHT is no exception from this rule. Therefore in closing, 
 In closing, Section \ref{sec:limitation} presents a summary and discussion of this chapter.
\justify
\section{Hilbert spectral analysis and the empirical mode decomposition}
\label{sec:emd}
As any new methods, the Hilbert Huang transform strives in improving upon previous signal processing paradigms. It rests on adaptivity, non linearity, non stationarity and physical meaningful time-frequency-energy paradigm. The latter is akin to the fundamental concept of uniqueness in the study of the solution of differential equations. 
In Fourier analysis, a time dependent signal $S(t)$, can be expressed as the following complex expansion
\begin{equation} \label{eq:fourier}
S(t) = \sum_{j=1}^{\infty}a_{j}\exp(i\omega_{j}t),
\end{equation}
where $a_{j}, \omega_{j}$ are constant amplitudes and frequencies respectively, and $i = \sqrt{-1}$. The Hilbert Huang transform generalizes the Fourier expansion (\ref{eq:fourier}) with time dependent amplitude $a(t)$ and frequency $\omega(t)$ as 
\begin{equation} \label{eq:hht}
S(t) = \sum_{j=1}^{n}a_{j}(t)\exp\left(i\int \omega_{j}(t)\mathrm{dt}\right).
\end{equation}
Equation (\ref{eq:hht}) expresses non stationarity as opposed to equation (\ref{eq:fourier}), by providing amplitude and frequency modulation. Here amplitude and frequency modulation refer to the time dependency of the amplitude $a(t)$ and the frequency $\omega(t)$. The former and the latter can be derive through the Hilbert transform. let $H(t)$ be the Hilbert transform of the signal $S(t)$ and given by
\begin{equation}\label{eq:hilbert}
H(t) = \frac{P}{\pi}\int_{-\infty}^{\infty}\frac{S(\tau)}{t-\tau}\mathrm{d\tau},
\end{equation}
where $P$ is the Cauchy principal value of the singular integral (\cite{huang08}). The amplitude $a(t)$ and the frequency $\omega(t)$
can hence be deduced by 
\begin{equation}\label{eq:amplitude}
a(t) = \sqrt{S(t)^{2} + H(t)^{2}}
\end{equation}

\begin{equation}\label{eq:frequency}
\omega(t) = \frac{d}{dt}\left(\tan^{-1}\left(\frac{H(t)}{S(t)} \right) \right)
\end{equation}
The amplitude and frequency modulation obtained from Hilbert transform in equation (\ref{eq:amplitude}, \ref{eq:frequency}) does not always provide a meaningful physical description of a signal (\cite{huang98}, \cite{huang08}).
 The empirical mode decomposition (EMD) on the other hand, decomposes a given signal into sub-components call intrinsic mode functions (IMFs), such that every IMF gives physically meaningful amplitude and frequency modulation, (\cite{huang98}).
 \justify
  For a target signal $S(t)$, the goal is to obtained its $n$ fundamental parts (IMFs), denoted here by $s_{j}, j =1,\cdots,n$, through the empirical mode decomposition, such that
\begin{equation}\label{eq:emd-decomposition}
S(t) = \sum_{j=1}^{n}s_{j}(t) + r(t),
\end{equation}
where $r(t)$ is the residual, which is either a constant or a monotone function. The derivation of the intrinsic mode functions $s_{j}$, by means of the empirical mode decomposition, relies on the following key assumptions.
\begin{definition}\label{def:emd}
An intrinsic mode function (IMF) must satisfy the following conditions:
\begin{enumerate}
\item The number of extrema and the number of zero crossing must either equal or differ by one 
\item At any data point, the mean value of the envelope defined using the local maxima and the envelope defined by using the local minima is zero.
\end{enumerate}
\end{definition}
The empirical mode decomposition is an iterative algorithm that generates intrinsic mode functions, each satisfying definition \ref{def:emd}.
\justify
The empirical mode decomposition can be described as follow:
\begin{enumerate}
	\item Compute the upper and the lower envelope curve of the target signal $S(t)$
	\item At in the first iteration $i=1$, compute the mean $m_{i}(t)$ between the upper and the lower envelope cure of $S(t)$
	\item Compute the fist proto IMF as 
	\begin{equation}\label{eq:proto}
		h_{i1}(t) = S(t)-m_{i}(t).
	\end{equation}
	If $h_{i1}$ satisfies definition \ref{def:emd}, then it is an IMF and it is denoted by 
	\begin{equation}\label{eq:imf}
		h_{i1}(t) = c_{i}(t).
	\end{equation}
	otherwise
	
	\item Set $h_{i1}(t)$ as the input signal and repeat set 1,2 and 3 $k$ time, until $h_{ik}(t)$ satisfies definition \ref{def:emd}.
	\item After finding the first IMF, set the first IMF as input signal and repeat step 1,2,3 and 4 to obtain the remaining IMFs. If an IMF is monotone, set it as the residual and you are done.
\end{enumerate}
Figure \ref{fig:emd3} shows the the signal  $S(t) = \sin(10 \pi t) + \sin(20 \pi t) $ (top) and its IMFs as well as the residual. In most function approximation methods, a set of basis functions $\varphi_{j}(t), j = 1,\cdots,n$ coupled with coefficients $a_{j}$, are used to decompose a function $f(t)$ as 
\begin{equation}\label{eq:basis}
	f(t) = \sum_{j=1}^{n}a_{j}\varphi_{j}(t).
\end{equation}
However, the Hilbert Huang transform will decompose $f(t)$ as 
\begin{equation}\label{eq:hht1}
f(t) = \sum_{j=1}^{n}c_{j}(t) + r(t).
\end{equation}
In equation (\ref{eq:basis}) the basis $\varphi_{j}(t)$ are chosen before hand (a-priory), and the coefficients are computed through an integral or summation operation. A sort of \say{bias} is imposed on the the function $f(t)$. On the hand, in equation (\ref{eq:hht1}) one can clearly state that the basis function are directly computed based on the function $f(t)$. This illustrates the concept of adaptivity, central to the Hilbert Huang transform.
\begin{figure}[H] %  figure placement: here, top, bottom, or page
   \centering
   \includegraphics[width=6in]{../fig/imfEMD.png} 
   \caption{The intrinsic mode functions (IMF ) and the residual (in red) generated from the input signal $S(t) = \sin(10 \pi t) + \sin(20 \pi t) $}
   \label{fig:emd3}
\end{figure}
\justify
The intrinsic mode functions obtained through the EMD process constitute an adaptive basis that satisfies the mathematical properties of convergence, completeness, orthogonality and uniqueness, \cite{huang98}. Furthermore, if $a_{j}(t)$ and $\omega_{j}(t)$ are amplitude and frequency modulation corresponding to IMF $j$, then the original signal $S(t)$ can also be recover as 
\begin{equation}\label{eq:recover2}
S(t) = \Re{\left( \sum_{j=1}^{n}a_{j}(t)\exp\left(i\int\omega_{j}(t)\mathrm{dt}\right)  \right)},
\end{equation} 
where the symbol $\Re(\cdot)$ represents the real part of the expression its encompasses, $i=\sqrt{-1}$, and $n$ is the total number of IMFs obtained from decomposing a signal $S(t)$. Recall that the amplitudes $a_{j}(t)$ and the frequencies $\omega_{j}(t)$ can be computed through the Hilbert transform from equations (\ref{eq:hilbert}, \ref{eq:amplitude}, and \ref{eq:frequency}). An analog representation of equation (\ref{eq:recover2}) in terms of Fourier expansion would be 
\begin{equation}\label{eq:recoverFourier}
S(t) = \Re{\left( \sum_{j=1}^{n}a_{j}\exp\left(i\omega_{j}\right)  \right)},
\end{equation} 
where this time, the amplitude $a_{j}$ and the frequency $\omega_{j}$ are constant. The Hilbert Huang transform (HHT) offers two different approaches to recover a decomposed signal. The first one is described by equation (\ref{eq:hht1}) and include the IMFs, while the second approach is given by equation (\ref{eq:recover2}) and includes the instantaneous amplitude and the instantaneous frequency. This gives the HHT a leverage over Fourier transform in nonlinear an non stationary data analysis.
%%%%%%%%%%%%%%%%%%%%%%%%%%%%%%%%%%%%%%%%%%%%%%%%%%%%%%%%%%%%%%%%%%%%%%%%%%%%%%%%%%%%%
%%%%%%%%%%%%%%%%%%%%%%%%%%%%%%%%%%%%%%%%%%%%%%%%%%%%%%%%%%%%%%%%%%%%%%%%%%%%%%%%%%%%%
%%%%%%%%%%%%%%%%%%%%%%%%%%%%%%%%%%%%%%%%%%%%%%%%%%%%%%%%%%%%%%%%%%%%%%%%%%%%%%%%%%%%%

\section{Application of the Empirical mode decomposition to pulse detection}
\label{sec:pulse}
After introducing the Hilbert Huang transform in the previous section, we are now ready to apply the empirical mode decomposition to a case study: Detecting pulses emitted by a bearing with crack in the outer ring. First of all, a few recalls: Figure \ref{fig:bearing-architecture} shows a geometry of a bearing, with different parts.
\begin{figure}[H] %  figure placement: here, top, bottom, or page
   \centering
   \includegraphics[width=4in]{../fig/bearing.png} 
   \caption{Geometrical representation of a bearing}
   \label{fig:bearing-architecture}
\end{figure}
\justify
A fault occurring at the outer ring is called a ball pass frequency outer ring (BPFO) defect. The frequency (in Hz) at which it occurs can be approximated as followed:
\begin{equation}
BPFO = \frac{np}{2}S\left(1-\frac{BD}{PD}\cos\left(\beta\right)  \right) \nonumber,
\end{equation}
where S is the rotating speed of the motor on which the bearing is attached. nb is the number of rolling elements (balls), BD is a rolling element diameter. The pitch diameter PD is half the hight of the inner ring, and the contact angle $\beta$ is the angle formed when a rolling element touches the cage. The data and the experimental setup used for this case study were described in chapter 2. In this experiment, four bearings are mounted on a motor rotating at 2000 rotations per minutes. The motor runes until bearing number 1 was severely affected by a ball pass frequency outer race defect. The data consists of 984 samples. Each sample contains  acceleration measurements consisted of 20 480 data points, obtained by mounting an accelerometer on each bearing. Recall that an accelerometer is a sensor used to quantify acceleration when a system vibrates. After this background information, let describe the methodology used to detect pulses emitted by a bearing with cracks in the outer ring.
\justify
The method consists of applying the empirical mode decomposition, followed by a robust seasonal trend decomposition method called STL, which stands for Seasonal Trend decomposition procedure based on LOESS,  (\cite{Cleveland-et-al-1990}). LOESS is the abbreviation of Locally Estimated Scatterplot Smoother, and is a nonparametric curve fitting procedure. It can be used for \say{Data exploration, diagnostic checking of parametric models and provides a nonparametric regression surface}, (\cite{Cleveland-1979}, \cite{Cleveland-et-al-1988}).
\justify
\subsection{Seasonal Trend decomposition based on Loess (STL)}
The STL sequentially applies the locally estimated scatterplot smoother (LOESS) in order to obtain cyclical and trend components of a signal. Here a cyclical component is a periodically occurring pattern in a signal.
If $S(t)$ is a signal, the goal is to obtain the decomposition 
\begin{equation}
S(t) = T(t) + C(t) + R(t),
\end{equation} 
where $T(t)$ an $C(t)$ are the trend and cyclical components, and $R(t)$ is the residual obtained from subtracting  the trend and the cyclical component from the signal. The key ingredient in the STL, is the locally estimated scatterplot smoother procedure. The latter being a non parametric regression method, does not rely on any a-priory assumption on the shape of the curve that needs to be fitted. It therefore provides a flexible approach to curve fitting, by capturing local, as well as global characteristics of a signal. Since pulses emitted by a bearing roller elements passing a crack, are local periodic events on a global scale, the STL provides the necessary mathematical apparatus to capture them.
\justify
By design, the STL is computational efficient, robust in the sense that it uses a robust statistical estimator, and deals very well with missing and distorted data points (\cite{Cleveland-et-al-1990}). In the subsequent paragraphs, an account of the LOESS procedure is presented, followed by the description of the STL.
\justify
Given a signal represented by a set of $n$ points, conveniently expressed by their coordinates
\begin{equation}
\{ (t_{j}, y_{j}), \quad j = 1, \cdots, n\},
\end{equation}
where $t_{j}$ is the independent variable on the time scale, and $y_{j}$ is the dependent variable. The LOESS procedure finds a smooth function $\hat{g}(t)$, defined everywhere on the scale of the independent variable, such that the estimate of $y_{j}$ denoted by $\hat{y}_{j}$ at $t_{j}$ is computed as 
\begin{equation}
\begin{split}
\hat{y}_{j} &= \hat{g}(t_{j}) \\
                &=\sum_{k=0}^{d}\hat{\alpha}_{k}(t_{j})t_{j}^{k}
\end{split}
\end{equation}
where the $\hat{\alpha}_{k}(t_{j})$ are the values $\alpha_{k}$ that minimize 
\begin{equation}
\sum_{k=1}^{n}w_{k}(t_{j})\left(y_{k}-\alpha_{0}-\alpha_{1}t_{k}-\cdots -\alpha_{d}t_{k}^{d}\right)^{2}
\end{equation}
with 
\begin{equation}
w_{k}(t_{j}) = W\left( \frac{t_{k}-t_{j}}{h_{j}} \right)
\end{equation}

\begin{equation}
W(t) = 
  \begin{cases}
  \left(1-t^{3}\right)^{3} &\quad\text{for } 0\leq t \le 1\\
   0 &\quad\text{for} t\geq 1
 \end{cases}
\end{equation}
and $h_{j}$ is the distance from $t_{j}$ to the nearest neighbor of $t_{j}$. For a signal $S(t)$, by setting 
\begin{equation}
S(t) = T(t) + C(t) + R(t), \nonumber
\end{equation}  
and by applying sequentially the LOESS procedure, the STL is able to extract the trend $T(t)$ and the cyclical component $C(t)$. A detail treatment on how the trend and cyclical components are extracted can be found in  (\cite{Cleveland-et-al-1990}).




%%%%%%%%%%%%%%%%%%%%%%%%%%%%%%%%%%%%%%%%%%%%%%%%%%%%%%%%%%%%%%%%%%%%%%%%%%%%%%%%%%%%%%%%%%%%%%
%%%%%%%%%%%%%%%%%%%%%%%%%%%%%%%%%%%%%%%%%%%%%%%%%%%%%%%%%%%%%%%%%%%%%%%%%%%%%%%%%%%%%%%%%%%%%%

\subsection{Results from applying empirical mode decomposition and seasonal trend decomposition base on Loess}
In this section we apply the empirical mode decomposition followed by the seasonal trend decomposition based on LOESS to extract pulses emitted by a failed bearing.
Figure \ref{fig:pulse} shows a flow diagram, describing the methodology to obtain pulses emitted from a bearing with ball pass frequency outer race defect. An input signal goes through the empirical mode decomposition to generate intrinsic mode functions. Furthermore, the STL is applied on the intrinsic mode functions to generate the pulses. 
The outer race ball pass frequency defect, can be viewed as a crack located at the outer ring of a bearing. As the bearings roller elements or balls pass the crack, we should expect to see periodic high frequency pulses in the signal. 

%
\begin{figure}[H]
\begin{tikzpicture}
  [node distance=.8cm,
  start chain=going below,]
     \node[punktchain, join] (intro) {\textcolor{blue}{Input signal}};
     \node[punktchain, join] (probf)      {\textcolor{violet}{EMD}};
     \node[punktchain, join] (investeringer)      {\textcolor{blue}{IMFs}};
     \node[punktchain, join] (perfekt) {\textcolor{violet}{STL}};
     \node[punktchain, join] (perfekt) {\textcolor{blue}{Pulse signal}};
     %\node[punktchain, join, ] (emperi) {Fast Fourier transform};
     % \node (asym) [punktchain ]  {Asymmetrisk information};
      \begin{scope}[start branch=venstre,
        %We need to redefine the join-style to have the -> turn out right
        every join/.style={->, thick, shorten <=1pt}, ]
        \node[punktchain, on chain=going left, join=by {->}] (risiko) {\textcolor{blue}{Periodogram}};
      \end{scope}
      \begin{scope}[start branch=hoejre,]
      %\node (finans) [punktchain, on chain=going right] {Det finansielle system};
    \end{scope}
    \end{tikzpicture}
  \caption{Schematic description of the methodology to obtaining a pulse like signal containing all diagnostic information for bearing fault detection}
   \label{fig:pulse}
\end{figure}
\justify
The STL process extracts the most significant and relevant features or information from the intrinsic mode functions. The result is a signal with strong sinusoidal components. The signal obtained from the STL process is called pulse signal (a name derived in this thesis). This is due to the fact that the signal (bottom) in Figure \ref{fig:pulse} obtained from applying the STL process on an intrinsic mode function (middle), has a pulse like representation. This can be conceptually seen as the pulse emitted when a defect is stricken in a bearing. 
\begin{figure}[H] %  figure placement: here, top, bottom, or page
   \centering
   \includegraphics[width=6in]{../fig/emd-stl.png} 
   \caption{ A pulse signal extracted by applying EMD followed by STL. The top graph represents the vibration time signal. The middle graph is the fifth intrinsic mode function. The bottom graph is the signal resulting from applying the STL on the IMF. The pulses represents the periodic high frequency signal emitted by a bearing when the roller elements pass a crack located in the outer ring.}
   \label{fig:emd-stl}
\end{figure}
\justify
Figure \ref{fig:emd-stl} shows, the result from applying the EMD and the STL to an input signal. The top graph represents the input vibration signal obtained by mounting an accelerometer on the defected bearing. The signal is measured after 6020 minutes from the start of the experiment. Recall that, after 9840 minutes, the defect endured by the bearing becomes severe, and the experiment is stoped. The middle graph of Figure \ref{fig:emd-stl} shows the fifth IMF obtained from applying the EMD to the input signal. The IMF exhibits the presence of high frequency patterns.

\subsubsection{Ball pass frequency outer race detection} 
%The middle graph shows the fifth intrinsic mode function generated from applying the EMD on the input signal. This IMF shows patterns.
%This is precisely what is require to capture a pulse, which can be viewed as a periodic emanation of an event such as a bearing ball passing a crack.
 \begin{figure}[H]
 	\centering
 	\includegraphics[width=0.8\linewidth]{../fig/periodogram_bpfo/start_imf1_bpfo}
 	\includegraphics[width=0.8\linewidth]{../fig/periodogram_bpfo/end_imf1_bpfo}
 	\caption{Periodogram of the pulse signal obtained from the first IMF, at the beginning (top) and the end (bottom) of experiment number 2}
 	\label{fig:startimf1bpfo}
 \end{figure}
\justify
Figure \ref{fig:startimf1bpfo} shows the periodogram of the pulse signal of the first IMF obtained at the beginning (top) and at the end of experiment number 2. The shaft frequency (33 Hz) and its first and second harmonics are clearly visible.
The ball pass outer race frequency defect (236.4 Hz) and its first harmonic are also visible. An increase in energy is also noticeable at the end of the experiment, suggesting that the bearing degradation is pronounced.
\justify
The difference between the theoretical value of the ball pass outer frequency defect (236.4 Hz) and the one on the periodogram (236.03 Hz) is about 0.16$\%$. While the difference between the theoretical shaft frequency (33.33 Hz) and the on the periodogram (33.63 Hz) is about 0.9 $\%$. This shows that the proposed method is quit accurate in detecting important diagnosis information from the vibration signal.
\justify
In general, the pulse signals extracted from the intrinsic mode functions through the STL method, had more diagnosis information for the fist two IMFs. This can been seen in Figure \ref{fig:startimf2bpfo}, which shows the periodogram of the pulse signals extracted from the second intrinsic mode function, at the beginning and at the end of experiment 2. At the end of the experiment, the energy of the ball pass outer race defect and its harmonic increased.
\justify
The highest intrinsic mode functions contain mostly the shaft rotation speed and sometimes its harmonics. This can been seen in Figure \ref{fig:startimf8shaft} which shows the Peridogram of the pulse signals obtained from IMF number 8, at the beginning (top) and at the end (bottom) of experiment number 2.

 \begin{figure}[H]
	\centering
	\includegraphics[width=0.8\linewidth]{../fig/periodogram_bpfo/start_imf2_bpfo}
	\includegraphics[width=0.8\linewidth]{../fig/periodogram_bpfo/end_imf2_bpfo}
	\caption{Periodogram of the pulse signal obtained from the second IMF, at the beginning (top) and the end (bottom) of experiment number 2.}
	\label{fig:startimf2bpfo}
\end{figure}

 \begin{figure}[H]
	\centering
	\includegraphics[width=0.8\linewidth]{../fig/periodogram_bpfo/start_imf8_shaft}
	\includegraphics[width=0.8\linewidth]{../fig/periodogram_bpfo/end_imf8_shaft}
	\caption{Periodogram of the pulse signal obtained from IMF number 8, at the beginning (top) and the end (bottom) of experiment number 2.}
	\label{fig:startimf8shaft}
\end{figure}
\justify
\subsubsection{ Ball pass inner race frequency detection }

\begin{figure}[H]
	\centering
	\includegraphics[width=0.8\linewidth]{../fig/periodogram_bpfi/start_imf1_bpfi}
	\includegraphics[width=0.8\linewidth]{../fig/periodogram_bpfi/end_imf1_bpfi}
	\caption{Periodogram of the pulse signal obtained from IMF number 1, at the beginning (top) and the end (bottom) of experiment number 1.}
	\label{fig:startimf1bpfi}
\end{figure}
\justify
Figure \ref{fig:startimf1bpfi} shows the periodograms obtained from the pulse signals of the first intrinsic mode functions at the start (top) and at the end (bottom) of experiment number 1. The inner race defect frequency is visible in both cases.
However, the shaft frequency is not visible.
\section{Summary and discussion}
\label{sec:limitation}
The proposed method consists of transforming a vibration signal to a pulse like signal, which contains all diagnostics information in terms of bearing failure detection. In particular, the outer race and the inner race defect. The present method uses the empirical mode decomposition (EMD), followed by the seasonal trend decomposition method based on Loess (STL). The former is a signal decomposition method suited for non stationary and non linear signal, while the letter extracts periodic components from a signal.
\justify
Together they are able to generate a noise less periodogram which contains relevant frequency information for bearing fault detection. The periodogram approximates the power spectral density (PSD) of a signal. The PSD is the energy distribution of frequencies components of a signal.
By applying the proposed method, it was possible to identify conspicuous outer and inner race defect frequencies, in a periodogram pertaining to a bearing undergoing failure. In addition, the periodogram exhibited clearly the rotational frequency of the machine shaft.








\blankpage
\end{document}

