% !TEX encoding = UTF-8 Unicode
%!TEX root = ../Main/thesis.tex
% !TEX spellcheck = en-US
%%=========================================
\documentclass[../Main/thesis.tex]{subfiles}
\begin{document}
\chapter[Hilbert Huang transform for high frequency pulse detection]{Hilbert Huang transform for high frequency pulse detection}
\label{sec:hht}

Most signal analysis methods such as Fourier transform, impose an a-priory basis functions on the signal to be analyze. In the case of Fourier analysis, the basis functions are trigonometric extensions. Although this implies a rigorous mathematical treatment, the resulting signal decomposition is limited by the mathematical assumptions. Two such assumptions are linearity and stationarity. As most phenomenon in nature are non linear and non stationary, this mathematical approach, although rigorous, lack an important property: Adaptivity. The latter refers to capturing the intrinsic properties of a signal, without imposing an a-priory basis function, \cite{huang98}, \cite{huang08}. 
\justify 
The Hilbert Huang transform (HHT) was precisely developed to deal with non linear and non stationary processes, in an adaptive fashion. It combines Hilbert spectral analysis with the so call empirical mode decomposition (EMD), to adaptively decompose a signal into its fundamental components called intrinsic mode functions (IMF), \cite{huang98}.
The richness of the HHT span from the analysis of differential equations to the study of geophysical phenomenons. It has been successful applied to the analysis of solutions of nonlinear differential equations such as the Duffing and the Lorentz equations. The intrinsic frequency, the forcing function and the low intensity subharmonics of the numerical solution for the nonlinear Duffing equation has been extracted through the EMD process, \cite{huang98}. The decomposition of the solution of the Lorentz equation, revealed transient components with different frequencies and damping characteristics, \cite{huang98}. It has been applied to tracking seismic wave propagation by identifying high and low frequency seismic wave [Vasudevan and Cook 2000, Zhan et al, 2003a, 3003b, zhang 2000]. Its application to the 1999 Taiwan earthquake, revealed that The Fourier transform underrepresented low frequency energy because of its linear characteristics [Huang et al 2001]. In this chapter, we apply the HHT in bearing fault analysis.
\justify
In section \ref{sec:emd}, we start by presenting the empirical mode decomposition (EMD), which is the back bone in decomposing a signal adaptively. In section \ref{sec:pulse}, We show that the HHT is able to recover the high frequency pulse emitted by a bearing with a crack. Lastly we outline some of the limitations of the HHT based on the empirical mode decomposition.
\justify
\section{The empirical mode decomposition and the Hilbert spectral analysis}
\label{sec:emd}

\section{Application of the Hilbert Huang transform to pulse detection }
\label{sec:pulse}



\blankpage
\end{document}

