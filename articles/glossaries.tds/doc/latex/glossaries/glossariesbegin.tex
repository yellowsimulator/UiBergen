\documentclass{nlctdoc}

\usepackage[utf8]{inputenc}
\ifpdf
\usepackage[T1]{fontenc}
\usepackage{mathpazo}
\usepackage[scaled=.88]{helvet}
\usepackage{courier}
\usepackage{xr-hyper}
\fi

\usepackage{alltt}
\usepackage{pifont}
\usepackage{array}
\usepackage{color}

\usepackage[colorlinks,linkcolor=blue,pdfauthor={Nicola L.C. Talbot},
            pdftitle={The glossaries package: a guide for beginners}]{hyperref}
\usepackage[nomain]{glossaries}

\newcommand*{\yes}{\ding{52}}
\newcommand*{\no}{\ding{56}}

\newcommand*{\proyes}{\textcolor{green}{\yes}}
\newcommand*{\prono}{\textcolor{red}{\no}}

\newcommand*{\conyes}{\textcolor{red}{\yes}}
\newcommand*{\conno}{\textcolor{green}{\no}}

\newcommand*{\code}[1]{%
 \texorpdfstring{{\frenchspacing\obeyspaces\ttfamily#1}}{#1}}
\newcommand*{\file}[1]{\texorpdfstring{\texttt{#1}}{#1}}

\title{The glossaries package v4.42: 
a guide for beginners}
\author{Nicola L.C. Talbot}
\date{2019-01-06}

\ifpdf
  \externaldocument{glossaries-user}
\fi

\makeatletter
\newcommand*{\optionlabel}[1]{%
 \@glstarget{option#1}{}\textbf{Option~#1}}
\makeatother

\newcommand*{\opt}[1]{\hyperlink{option#1}{Option~#1}}
\newcommand*{\optsor}[2]{Options~\hyperlink{option#1}{#1}
or~\hyperlink{option#2}{#2}}
\newcommand*{\optsand}[2]{Options~\hyperlink{option#1}{#1}
and~\hyperlink{option#2}{#2}}

\begin{document}
\maketitle

\begin{abstract}
The \styfmt{glossaries} package is very flexible, but this means
that it has a~lot of options, and since a user guide is supposed to
provide a complete list of all the high-level user commands, the main
user manual is quite big. This can be rather
daunting for beginners, so this document is a brief introduction
just to help get you started. If you find yourself saying, \qt{Yeah,
but how can I do\ldots?} then it's time to move on to the 
\docref{main user manual}{glossaries-user}.

I've made some statements in this document that don't actually tell
you the full truth, but it would clutter the document and cause
confusion if I keep writing \qt{except when \ldots} or \qt{but you can
also do this, that or the other} or \qt{you can do it this way but
you can also do it that way, but that way may cause complications
under certain circumstances}.
\end{abstract}

\tableofcontents

\section{Getting Started}
\label{sec:start}

As with all packages, you need to load \styfmt{glossaries} with
\cs{usepackage}, but there are certain packages that must be loaded
before \styfmt{glossaries}, \emph{if} they are required: \sty{hyperref},
\sty{babel}, \sty{polyglossia}, \sty{inputenc} and \sty{fontenc}.
(You don't have to load these packages, but if you want them, you
must load them before \styfmt{glossaries}.)

\begin{important}
If you require multilingual support you must also install 
the relevant language module. Each language module is called
\file{glossaries-}\meta{language}, where \meta{language} is the
root language name. For example, \file{glossaries-french}
or \file{glossaries-german}. If a language module is required,
the \styfmt{glossaries} package will automatically try to load it
and will give a warning if the module isn't found.
\end{important}

Once you have loaded \styfmt{glossaries}, you need to define
your terms in the preamble and then you can use them throughout the
document. Here's a simple example:
\begin{verbatim}
 \documentclass{article}

 \usepackage{glossaries}

 \newglossaryentry{ex}{name={sample},description={an example}}

 \begin{document}
 Here's my \gls{ex} term.
 \end{document}
\end{verbatim}
This produces:
\begin{quote}
Here's my sample term.
\end{quote}
Here's another example:
\begin{verbatim}
 \documentclass{article}

 \usepackage{glossaries}

 \setacronymstyle{long-short}

 \newacronym{svm}{SVM}{support vector machine}

 \begin{document}
 First use: \gls{svm}. Second use: \gls{svm}.
 \end{document}
\end{verbatim}
This produces:
\begin{quote}
First use: support vector machine (SVM). Second use: SVM.
\end{quote}
In this case, the text produced by \verb|\gls{svm}| changed after the first
use. The first use produced the long form followed by the short form
in parentheses because I set the acronym style to
\code{long-short}. I suggest you try the above two examples to
make sure you have the package correctly installed.
If you get an \code{undefined control sequence} error, check that
the version number at the top of this document matches the version
you have installed. (Open the \file{.log} file and search for the
line that starts with \code{Package: glossaries} followed by a
date and version.) 

\begin{important}
Be careful of fragile commands. If a command causes a problem when
used in one of the \cs{newglossaryentry} fields, consider adding
\cs{glsnoexpandfields} before you start defining your entries.
\end{important}

Abbreviations are slightly different if you use the extension package
\sty{glossaries-extra} (which needs to be installed separately):
\begin{verbatim}
 \documentclass{article}

 \usepackage{glossaries-extra}

 \setabbreviationstyle{long-short}% glossaries-extra.sty

 \newabbreviation{svm}{SVM}{support vector machine}% glossaries-extra.sty

 \begin{document}
 First use: \gls{svm}. Second use: \gls{svm}.
 \end{document}
\end{verbatim}
If you still want to use \cs{newacronym} (rather than
\cs{newabbreviation}) then you need the optional
argument of \cs{setabbreviationstyle}:
\begin{verbatim}
 \documentclass{article}

 \usepackage{glossaries-extra}

 \setabbreviationstyle[acronym]{long-short}% glossaries-extra.sty only

 \newacronym{svm}{SVM}{support vector machine}

 \begin{document}
 First use: \gls{svm}. Second use: \gls{svm}.
 \end{document}
\end{verbatim}
\begin{important}
You can't use \cs{setacronymstyle} with \sty{glossaries-extra}.
\end{important}

If you like, you can put all your definitions in another file (for
example, \file{defns.tex}) and load that file in the preamble using
\cs{loadglsentries} with the filename as the argument. For example:
\begin{verbatim}
\loadglsentries{defns}
\end{verbatim}
If you find you have a really large number of definitions that are
hard to manage in a \file{.tex} file, you might want to have a
look at \app{bib2gls} (installed separately) which requires 
a \file{.bib} format instead that can be managed by an application
such as JabRef.

Don't try inserting formatting commands into the definitions as they
can interfere with the underlying mechanism. Instead, the
formatting should be done by the style. For example, suppose I~want
to replace \code{SVM} with \verb|\textsc{svm}|, then
I~need to select a style that uses \cs{textsc}, like this (for the
base \styfmt{glossaries} style):
\begin{verbatim}
 \documentclass{article}

 \usepackage{glossaries}

 \setacronymstyle{long-sc-short}

 \newacronym{svm}{svm}{support vector machine}

 \begin{document}
 First use: \gls{svm}. Second use: \gls{svm}.
 \end{document}
\end{verbatim}
The abbreviation styles have a different naming scheme with
\sty{glossaries-extra}:
\begin{verbatim}
 \documentclass{article}

 \usepackage{glossaries-extra}

 \setabbreviationstyle{long-short-sc}% glossaries-extra.sty

 \newabbreviation{svm}{svm}{support vector machine}% glossaries-extra.sty

 \begin{document}
 First use: \gls{svm}. Second use: \gls{svm}.
 \end{document}
\end{verbatim}
With \sty{glossaries-extra} you can have multiple abbreviation
styles for different categories. Many of these styles have their
own associated formatting commands that can be redefined 
for minor adjustments. For example:
\begin{verbatim}
 \documentclass{article}

 \usepackage{glossaries-extra}

 \setabbreviationstyle[statistical]{long-short-sc}
 \setabbreviationstyle[bacteria]{long-only-short-only}

 % Formatting commands used by 'long-only-short-only' style:
 \renewcommand*{\glsabbrvonlyfont}[1]{\emph{#1}}
 \renewcommand*{\glslongonlyfont}[1]{\emph{#1}}

 % Formatting command used by 'long-short-sc' style:
 % (make sure abbreviation is converted to lower case)
 \renewcommand*{\glsabbrvscfont}[1]{\textsc{\MakeLowercase{#1}}}

 \newabbreviation
  [
    category={statistical}% glossaries-extra.sty key
  ]
  {svm}{SVM}{support vector machine}

 \newabbreviation
  [
    category={bacteria}% glossaries-extra.sty key
  ]
  {cbot}{C.~botulinum}{Clostridium botulinum}

 \begin{document}
 First use: \gls{svm}, \gls{cbot}.

 Next use: \gls{svm}, \gls{cbot}.
 \end{document}
\end{verbatim}
This produces:
\begin{quote}
First use: support vector machine (\textsc{svm}), \emph{Clostridium
botulinum}.

Next use: \textsc{svm}, \emph{C.~botulinum}.
\end{quote}

As you can hopefully see from the above examples, there are two main ways of
defining a term: as a general entry (\cs{newglossaryentry}) or as an
abbreviation (\cs{newacronym} or, with \sty{glossaries-extra},
\cs{newabbreviation}).

Regardless of the method of defining a term, the term is always 
given a label. In the first example, the label was
\code{ex} and in the other examples the label was \code{svm}
(and \code{cbot} in the last example). The
label is used to uniquely identify the term (like the standard
\cs{label}\slash\cs{ref} or \cs{cite} mechanism). It's best to just use the
following alphanumerics in the labels: \code{a}, \ldots,
\code{z}, \code{A}, \ldots, \code{Z}, \code{0}, \ldots,
\code{9}.  Sometimes you can also use punctuation characters but
not if another package (such as \sty{babel}) meddles with them.
Don't try using any characters outside of the basic Latin set with
\sty{inputenc} (for example, \'e or \ss) or things will go horribly
wrong. This warning only applies to the label. It doesn't apply to
the text that appears in the document.

\begin{important}
Don't use \cs{gls} in chapter or section headings as it can have
some unpleasant side-effects. Instead use \cs{glsentrytext} for
regular entries and one of \cs{glsentryshort}, \cs{glsentrylong}
or \cs{glsentryfull} for acronyms. Alternatively use
\sty{glossaries-extra} which provides special commands for use in
section headings, such as \cs{glsfmtshort}\marg{label}.
\end{important}

The above examples are reasonably straightforward. The difficulty
comes if you want to display a \emph{sorted} list of all the entries you
have used in the document. The \sty{glossaries-extra} package
provides a really easy way of listing all the defined entries:
\begin{verbatim}
 \documentclass{article}

 \usepackage[sort=none]{glossaries-extra}

 \newglossaryentry{potato}{name={potato},plural={potatoes},
  description={starchy tuber}}

 \newglossaryentry{cabbage}{name={cabbage},
  description={vegetable with thick green or purple leaves}}
 
 \newglossaryentry{turnip}{name={turnip},
  description={round pale root vegetable}}

 \newglossaryentry{carrot}{name={carrot},
  description={orange root}}

 \begin{document}
 Chop the \gls{cabbage}, \glspl{potato} and \glspl{carrot}.

 \printunsrtglossaries % list all entries
 \end{document}
\end{verbatim}
However this method doesn't sort the entries (they're listed in
order of definition) and it will display all the defined entries,
regardless of whether or not you've used them all in the document,
so \qt{turnip} appears in the glossary even though there's
no \verb|\gls{turnip}| (or similar) in the document.

The \pkgopt[none]{sort} option isn't essential in this case (there's
no other sort option available for this document), but it prevents
the automatic construction of the sort value and so slightly
improves the document build time.

Note that this example document uses the same command
(\cs{printunsrtglossaries}) that's used with \app{bib2gls} (\opt4)
but with \app{bib2gls} you instead need to use the 
\pkgopt{record} package option and one or more instances of
\cs{GlsXtrLoadResources} in the preamble (see below).

Most users prefer to have an automatically sorted list that only
contains entries that have been used in the document.
The \styfmt{glossaries} package provides
three options: use \TeX\ to perform the sorting (\opt1); use
\app{makeindex} to perform the sorting (\opt2); use \app{xindy} to
perform the sorting (\opt3). The extension package \sty{glossaries-extra}
provides a fourth method: use \app{bib2gls} (\opt4).

The first option (using \TeX) is the simplest method, as it doesn't
require an external tool, but it's very inefficient and the sorting
is done according to the English alphabet. To use this method, add
\cs{makenoidxglossaries} to the preamble and put
\cs{printnoidxglossaries} at the place where you want your glossary.
For example:
\begin{verbatim}
 \documentclass{article}

 \usepackage{glossaries}

 \makenoidxglossaries % use TeX to sort

 \newglossaryentry{potato}{name={potato},plural={potatoes},
  description={starchy tuber}}

 \newglossaryentry{cabbage}{name={cabbage},
  description={vegetable with thick green or purple leaves}}

 \newglossaryentry{turnip}{name={turnip},
  description={round pale root vegetable}}

 \newglossaryentry{carrot}{name={carrot},
  description={orange root}}

 \begin{document}
 Chop the \gls{cabbage}, \glspl{potato} and \glspl{carrot}.

 \printnoidxglossaries
 \end{document}
\end{verbatim}

\begin{important}
The \cs{makenoidxglossaries} method is very slow, uses an ASCII
comparator and often breaks if there are commands in the
\code{name} key.
\end{important}

Try this out and run \LaTeX\ (or pdf\LaTeX) \emph{twice}. The first
run won't show the glossary. It will only appear on the second run.
This doesn't include \qt{turnip} in the glossary because that term
hasn't been used (with commands like \verb|\gls{turnip}|) in the document.

The glossary has a vertical gap between the \qt{carrot} term and the
\qt{potato} term. This is because the entries in the glossaries are
grouped according to their first letter. If you don't want this gap,
just add \pkgopt{nogroupskip} to the package options:
\begin{verbatim}
\usepackage[nogroupskip]{glossaries}
\end{verbatim}
or you may want to try out a style that shows the group headings:
\begin{verbatim}
\usepackage[style=indexgroup]{glossaries}
\end{verbatim}
If you try out this example you may also notice that the description
is followed by a full stop (period) and a number. The number is the
location in the document where the entry was used (page~1 in this
case), so you can lookup the term in the glossary and be directed to
the relevant pages. It may be that you don't want this
back-reference, in which case you can suppress it using the
\pkgopt{nonumberlist} package option:
\begin{verbatim}
\usepackage[nonumberlist]{glossaries}
\end{verbatim}
If you don't like the terminating full stop, you can suppress that
with the \pkgopt{nopostdot} package option:
\begin{verbatim}
\usepackage[nopostdot]{glossaries}
\end{verbatim}

If you tried out the previous example with \sty{glossaries-extra}
and \cs{printunsrtglossaries} the terminating full stop is missing
and there are no number lists. You can add the full stop back with
\begin{verbatim}
\usepackage[nopostdot=false]{glossaries-extra}
\end{verbatim}
or
\begin{verbatim}
\usepackage[postdot]{glossaries-extra}
\end{verbatim}
If you want the number lists then you need to use an indexing option.


You may have noticed that I've used another command in the above example:
\cs{glspl}. This displays the plural form. By default, this is just
the singular form with the letter \qt{s} appended, but in the case of
\qt{potato} I had to specify the correct plural using the
\code{plural} key.

As I mentioned earlier, using \TeX\ to sort the entries is the
simplest but least efficient method. If you have a large glossary or
if your terms contain non-Latin or extended Latin characters, then
you will have a much faster build time if you use \app{makeindex}
(\opt2) or \app{xindy} (\opt3) or \app{bib2gls} (\opt4). If
you are using extended Latin or non-Latin characters, then
\app{xindy} or \app{bib2gls} are the recommended methods.
These methods are described in more detail in
\sectionref{sec:printglossaries}.

The rest of this document briefly describes the main commands
provided by the \styfmt{glossaries} package. (Most of these are also
available with \sty{glossaries-extra} but may behave slightly
differently.)

\section{Defining Terms}
\label{sec:defterm}

When you use the \sty{glossaries} package, you need to define
glossary entries in the document preamble. These entries could be
a~word, phrase, abbreviation or symbol. They're usually accompanied by
a~description, which could be a short sentence or an in-depth
explanation that spans multiple paragraphs. The simplest method of
defining an entry is to use:
\begin{definition}
\begin{alltt}
\cs{newglossaryentry}\marg{label}
\verb|{|
  name=\marg{name},
  description=\marg{description},
  \meta{other options}
\verb|}|
\end{alltt}
\end{definition}
where \meta{label} is a unique label that identifies this entry.
(Don't include the angle brackets \meta{ }. They just indicate the parts of
the code you need to change when you use this command in your document.) 
The \meta{name} is the word, phrase or symbol you are defining,
and \meta{description} is the description to be displayed in the
glossary.

This command is a \qt{short} command, which means that
\meta{description} can't contain a~paragraph break. If you have
a~long description, you can instead use:
\begin{definition}
\begin{alltt}
\cs{longnewglossaryentry}\marg{label}
\verb|{|
  name=\marg{name},
  \meta{other options}
\verb|}|
\marg{description}
\end{alltt}
\end{definition}

Examples:
\begin{enumerate}
\item Define the term \qt{set} with the label \code{set}:
\begin{verbatim}
\newglossaryentry{set}
{
  name={set},
  description={a collection of objects}
}
\end{verbatim}

\item Define the symbol $\emptyset$ with the label
\code{emptyset}:
\begin{verbatim}
\newglossaryentry{emptyset}
{
  name={\ensuremath{\emptyset}},
  description={the empty set}
}
\end{verbatim}
(This will also need a \code{sort} key if you use \optsor13, see
below.)

\item Define the phrase \qt{Fish Age} with the label
\code{fishage}:
\begin{verbatim}
\longnewglossaryentry{fishage}
{name={Fish Age}}
{%
  A common name for the Devonian geologic period 
  spanning from the end of the Silurian Period to
  the beginning of the Carboniferous Period.

  This age was known for its remarkable variety of 
  fish species.
}
\end{verbatim}
(The percent character discards the end of line character that would
otherwise cause an unwanted space to appear at the start of the
description.)

\item Take care if the first letter is an extended Latin or
non-Latin character (either specified via a~command such as 
\verb|\'e| or explicitly via the \sty{inputenc} package such 
as \code{\'e}). This first letter must be placed in a~group:

\begin{verbatim}
\newglossaryentry{elite}
{
  name={{\'e}lite},
  description={select group or class}
}
\end{verbatim}
or
\begin{alltt}
\verb|\newglossaryentry{elite}|
\verb|{|
  name=\verb|{{|\'e\verb|}lite}|,
  description=\verb|{select group or class}|
\verb|}|
\end{alltt}
\end{enumerate}
(For further details, see the section \qt{UTF-8} of the 
\href{https://ctan.org/pkg/mfirstuc}{\styfmt{mfirstuc}} user manual.)

If you use \app{bib2gls} with \sty{glossaries-extra} then the
terms must be defined in a \file{.bib} file. For example:
\begin{alltt}
\% Encoding: UTF-8

@entry\{set,
  name=\{set\},
  description=\{a collection of objects\}
\}

@entry\{emptyset,
  name=\verb|{\ensuremath{\emptyset}}|,
  description=\{the empty set\}
\}

@entry\{fishage,
  name=\{Fish Age\},
  description=\{A common name for the Devonian geologic period 
  spanning from the end of the Silurian Period to
  the beginning of the Carboniferous Period.

  This age was known for its remarkable variety of 
  fish species.\}
\}

@entry\{elite,
  name=\verb|{{|\'e\verb|}lite}|,
  description=\verb|{select group or class}|
\}
\end{alltt}
(The \file{.bib} format doesn't allow spaces in labels so you can't
have \code{fish age} as the label, but you can have 
\code{fish-age}.)
This method requires the \sty{glossaries-extra}'s \pkgopt{record}
package option:
\begin{verbatim}
\usepackage[record]{glossaries-extra}
\end{verbatim}
and the \file{.bib} file is specified in the resource command. For
example, if the \file{.bib} file is called \file{entries.bib}
then put the following line in the document preamble:
\begin{verbatim}
\GlsXtrLoadResources[src={entries}]
\end{verbatim}
You can have a comma-separated list. For example, if you also
have entries defined in the file \file{entries2.bib}:
\begin{verbatim}
\GlsXtrLoadResources[src={entries,entries2.bib}]
\end{verbatim}

There are other keys you can use when you define an entry. For
example, the \code{name} key indicates how the term
should appear in the list of entries (glossary), but if the term should
appear differently when you reference it with \cs{gls}\marg{label} in 
the document, you need to use the \code{text} key as well.

For example:
\begin{verbatim}
\newglossaryentry{latinalph}
{
  name={Latin Alphabet},
  text={Latin alphabet},
  description={alphabet consisting of the letters 
  a, \ldots, z, A, \ldots, Z}
}
\end{verbatim}
This will appear in the text as \qt{Latin alphabet} but will be listed in
the glossary as \qt{Latin Alphabet}.
With \app{bib2gls} this entry is defined in the \file{.bib}
file as:
\begin{verbatim}
@entry{latinalph,
  name={Latin Alphabet},
  text={Latin alphabet},
  description={alphabet consisting of the letters 
  a, \ldots, z, A, \ldots, Z}
}
\end{verbatim}

Another commonly used key is \code{plural} for specifying the
plural of the term. This defaults to the value of the \code{text}
key with an \qt{s} appended, but if this is incorrect, just use the
\code{plural} key to override it:
\begin{verbatim}
\newglossaryentry{oesophagus}
{
  name={{\oe}sophagus},
  plural={{\oe}sophagi},
  description={canal from mouth to stomach}
}
\end{verbatim}
(Remember from earlier that the initial ligature \cs{oe} needs to
be grouped.)

Abbreviations can be defined using
\begin{definition}
\cs{newacronym}\oarg{options}\marg{label}\marg{short}\marg{long}
\end{definition}
where \meta{label} is the label (as per \cs{newglossaryentry}), 
\meta{short} is the short form and \meta{long} is the long form. 
For example, the
following defines an abbreviation:
\begin{verbatim}
\newacronym{svm}{SVM}{support vector machine}
\end{verbatim}
This internally uses \cs{newglossaryentry} to define an entry with
the label \code{svm}. By default, the \code{name} key is set to
\meta{short} (\qt{SVM} in the above example) and the
\code{description} key is set to \meta{long} (\qt{support vector
machine} in the above example). If, instead, you want to be able to
specify your own description you can do this using the optional
argument:
\begin{verbatim}
\newacronym
 [description={statistical pattern recognition technique}]
 {svm}{SVM}{support vector machine}
\end{verbatim}

Before you define your acronyms (or other types of abbreviations), you 
need to specify which style to use with
\begin{definition}
\cs{setacronymstyle}\marg{style name}
\end{definition}
where \meta{style name} is the name of the style. There are a number
of predefined styles, such as: \code{long-short} (on first use
display the long form with the short form in parentheses);
\code{short-long} (on first use display the short form with the
long form in parentheses); \code{long-short-desc} (like
\code{long-short} but you need to specify the description); or
\code{short-long-desc} (like \code{short-long} but you need to
specify the description). There are some other styles as well that
use \cs{textsc} to typeset the acronym or that use a footnote on
first use. See the main user guide for further details.

The \sty{glossaries-extra} package provides improved abbreviation
handling with \href{http://www.dickimaw-books.com/gallery/sample-abbr-styles.shtml}{a lot more predefined styles}.
With this extension package, abbreviations are defined using:
\begin{definition}
\cs{newabbreviation}\oarg{options}\marg{label}\marg{short}\marg{long}
\end{definition}
You can still use \cs{newacronym} but it's redefined to 
use the new abbreviation interface. The style must now be set using:
\begin{definition}
\cs{setabbreviationstyle}\oarg{category}\marg{style name}
\end{definition}
The default \meta{category} is \code{abbreviation}. If you use
\cs{newacronym} the category is \code{acronym}, which is why you
need to use the optional argument if you define abbreviations with
\cs{newacronym} when \sty{glossaries-extra} has been loaded:
\begin{alltt}
\cs{setabbreviationstyle}[acronym]\marg{style name}
\end{alltt}
If you use \app{bib2gls} then abbreviations are defined in the
\file{.bib} file in the format:
\begin{alltt}
@abbreviation\{\meta{label},
  long=\marg{long form},
  short=\marg{short form}
\}
\end{alltt}

The plural forms for abbreviations can be specified using the
\code{longplural} and \code{shortplural} keys. For example:
\begin{verbatim}
\newacronym
 [longplural={diagonal matrices}]
 {dm}{DM}{diagonal matrix}
\end{verbatim}
or (with \sty{glossaries-extra}):
\begin{verbatim}
\newabbreviation % glossaries-extra.sty
 [longplural={diagonal matrices}]
 {dm}{DM}{diagonal matrix}
\end{verbatim}
If omitted, the defaults are again obtained by appending an \qt{s} to
the singular versions. With \app{bib2gls}, the definition in the
\file{.bib} file is:
\begin{verbatim}
@abbreviation{dm,
  short={DM},
  long={diagonal matrix},
  longplural={diagonal matrices}
}
\end{verbatim}

It's also possible to have both a~name and a~corresponding symbol.
Just use the \code{name} key for the name and the \code{symbol} key
for the symbol. For example:
\begin{verbatim}
\newglossaryentry{emptyset}
{
  name={empty set},
  symbol={\ensuremath{\emptyset}},
  description={the set containing no elements}
}
\end{verbatim}
or with \app{bib2gls} the definition in the \file{.bib} file is:
\begin{verbatim}
@entry{emptyset,
  name={empty set},
  symbol={\ensuremath{\emptyset}},
  description={the set containing no elements}
}
\end{verbatim}

If you want the symbol in the \code{name} field then you must
supply a \code{sort} value with \optsand13 otherwise you'll end up
with errors from \TeX\ or \app{xindy}. With \opt2
(\app{makeindex}) it's not quite so important but you may find
the resulting order is a little odd. For example:
\begin{verbatim}
\newglossaryentry{emptyset}
{
  name={\ensuremath{\emptyset}},
  sort={empty set},
  description={the set containing no elements}
}
\end{verbatim}
This displays the symbol as $\emptyset$ but sorts according to
\qt{empty set}. You may want to consider using
\sty{glossaries-extra}'s \pkgopt{symbols} package option which
provides
\begin{definition}
\cs{glsxtrnewsymbol}\oarg{options}\marg{label}\marg{symbol}
\end{definition}
This internally uses \cs{newglossaryentry} but automatically sets
the \code{sort} key to the \meta{label}. For example:
\begin{verbatim}
\documentclass{article}

\usepackage[symbols]{glossaries-extra}

\makeglossaries

\glsxtrnewsymbol % requires glossaries-extra.sty 'symbols' option
 [description={the set containing no elements}]
 {emptyset}% label (and sort value)
 {\ensuremath{\emptyset}}% symbol

\begin{document}
\gls{emptyset}

\printglossaries
\end{document}
\end{verbatim}
Now the sort value is \qt{emptyset} rather than the previous
\qt{empty set}.

With \app{bib2gls} you can define this in the \file{.bib} file
as
\begin{verbatim}
@entry{emptyset,
  name={\ensuremath{\emptyset}},
  description={the set containing no elements}
}
\end{verbatim}
in which case \app{bib2gls} will try to interpret the \code{name} 
field to determine the sort value. Alternatively you can use:
\begin{verbatim}
@symbol{emptyset,
  name={\ensuremath{\emptyset}},
  description={the set containing no elements}
}
\end{verbatim}
which will use the label (\code{emptyset}) as the sort value.
(You don't need the \pkgopt{symbols} package option in this case,
unless you want a separate symbols list.) The corresponding document
(where the definition is in the file \file{entries.bib}) is now:
\begin{verbatim}
\documentclass{article}

\usepackage[record]{glossaries-extra}

\GlsXtrLoadResources[src=entries]

\begin{document}
\gls{emptyset}

\printunsrtglossaries
\end{document}
\end{verbatim}

\section{Using Entries}
\label{sec:useterm}

Once you have defined your entries, as described above, you can
reference them in your document. There are a~number of commands to
do this, but the most common one is:
\begin{definition}
\cs{gls}\marg{label}
\end{definition}
where \meta{label} is the label you assigned to the entry when you
defined it. For example, \verb|\gls{fishage}| will display \qt{Fish
Age} in the text (given the definition from the previous section).
If you are using \app{bib2gls} then this will display ?? (like
\cs{ref} and \cs{cite}) until \app{bib2gls} has created the
relevant files and \LaTeX\ is rerun.

If you are using the \sty{hyperref} package (remember to load it
before \styfmt{glossaries}) \cs{gls} will create a hyperlink to the
corresponding entry in the glossary. If you want to suppress the
hyperlink for a particular instance, use the starred form \cs{gls*}
for example, \verb|\gls*{fishage}|. The other commands described in
this section all have a similar starred form.

If the entry was defined as an acronym (using \cs{newacronym} with
\styfmt{glossaries} described above) or an abbreviation (using
\cs{newabbreviation} with \sty{glossaries-extra}), then \cs{gls}
will display the full form the first time it's used and just the
short form on subsequent use. For example, if the style is
set to \code{long-short}, then \verb|\gls{svm}| will display
\qt{support vector machine (SVM)} the first time it's used, but the
next occurrence of \verb|\gls{svm}| will just display \qt{SVM}.
(If you use \cs{newacronym} with \sty{glossaries-extra} the default
doesn't show the long form on first use. You'll need to change the
style first, as described earlier.)

If you want the plural form, you can use:
\begin{definition}
\cs{glspl}\marg{label}
\end{definition}
instead of \cs{gls}\marg{label}. For example, \verb|\glspl{set}|
displays \qt{sets}.

If the term appears at the start of a~sentence, you can convert the
first letter to upper case using:
\begin{definition}
\cs{Gls}\marg{label}
\end{definition}
for the singular form or
\begin{definition}
\cs{Glspl}\marg{label}
\end{definition}
for the plural form. For example:
\begin{verbatim}
\Glspl{set} are collections.
\end{verbatim}
produces \qt{Sets are collections}.

If you've specified a symbol using the \code{symbol} key, you can
display it using:
\begin{definition}
\cs{glssymbol}\marg{label}
\end{definition}

\section{Displaying a List of Entries}
\label{sec:printglossaries}

In \sectionref{sec:start}, I mentioned that there are three options
you can choose from to create an automatically sorted glossary with the base 
\styfmt{glossaries} package. These are also available with the
extension package \styfmt{glossaries-extra} along with a fourth
option. These four options are listed below in a little more detail.
\Tableref{tab:optionsp+c} summarises the main advantages and
disadvantages.  (There's a more detailed summary in the main
\styfmt{glossaries} user manual.) See also
\href{https://www.dickimaw-books.com/latex/buildglossaries/}{Incorporating makeglossaries or makeglossaries-lite or bib2gls into the document build}.

\begin{table}[htbp]
 \caption{Comparison of Glossary Options}
 \label{tab:optionsp+c}
 {%
 \centering
 \begin{tabular}{>{\raggedright}p{0.3\textwidth}cccc}
   & \bfseries \opt1 & \bfseries \opt2 & \bfseries \opt3 & \bfseries
\opt4\\
   Requires \sty{glossaries-extra}? &
   \conno & \conno & \conno & \conyes\\
   Requires an external application? &
   \conno & \conyes & \conyes & \conyes\\
   Requires Perl? &
   \conno & \conno & \conyes & \conno\\
   Requires Java? &
   \conno & \conno & \conno & \conyes\\
   Can sort extended Latin
   or non-Latin alphabets? &
   \prono & \prono & \proyes & \proyes\\
   Efficient sort algorithm? &
   \prono & \proyes & \proyes & \proyes\\
   Can use different sort methods for each glossary? &
   \proyes & \prono & \prono & \proyes\\
   Any problematic sort values? &
   \conyes & \conyes & \conyes & \conno\\
   Can form ranges in the location lists? &
   \prono & \proyes & \proyes & \proyes\\
   Can have non-standard locations? &
   \proyes & \prono & \proyes\textsuperscript{\textdagger} & \proyes
 \end{tabular}
 \par}\medskip\footnotesize\textsuperscript{\textdagger} Requires some setting up.
\end{table}


\begin{description}
\item[]\optionlabel1: 

 This is the simplest option but it's slow and if
 you want a sorted list, it doesn't work for non-Latin alphabets.
 The \code{name} mustn't contain commands (or, if it does, the
 \code{sort} value must be supplied) unless you 
 have the package option \pkgopt{sanitizesort} or \pkgopt[def]{sort} 
 or \pkgopt[use]{sort}.

  \begin{enumerate}
    \item Add \cs{makenoidxglossaries} to your preamble (before you
    start defining your entries, as described in
    \sectionref{sec:defterm}).

    \item Put
\begin{definition}
\cs{printnoidxglossary}[sort=\meta{order},\meta{other options}]
\end{definition}
    where you want your list of entries to appear. The sort
    \meta{order} may be one of: \code{word} (word ordering),
    \code{letter} (letter ordering), \code{case} (case-sensitive
    letter ordering), \code{def} (in order of definition) or
    \code{use} (in order of use). Alternatively, use
\begin{definition}
\cs{printnoidxglossaries}
\end{definition}
    to display all your glossaries (if you have more than one). This
    command doesn't have any arguments.

    This option allows you to have different sort methods. For
    example:
\begin{verbatim}
\printnoidxglossary[sort=word]% main glossary
\printnoidxglossary[type=symbols,sort=use]% symbols glossary
\end{verbatim}

    \item Run \LaTeX\ twice on your document. (As you would do to
    make a~table of contents appear.) For example, click twice on
    the \qt{typeset} or \qt{build} or \qt{PDF\LaTeX} button in your editor.
  \end{enumerate}
Here's a complete document (\file{myDoc.tex}):
\begin{verbatim}
\documentclass{article}

\usepackage{glossaries}

\makenoidxglossaries % use TeX to sort

\newglossaryentry{sample}{name={sample},
  description={an example}}

\begin{document}
A \gls{sample}.

\printnoidxglossaries % iterate over all indexed entries
\end{document}
\end{verbatim}
Document build:
\begin{verbatim}
pdflatex myDoc
pdflatex myDoc
\end{verbatim}

\item[]\optionlabel2:

   This option uses an application called \app{makeindex} to sort 
   the entries. This application comes with all modern \TeX\ distributions, 
   but it's hard-coded for the non-extended Latin alphabet. This process 
   involves making \LaTeX\ write the glossary information to a temporary 
   file which \app{makeindex} reads. Then \app{makeindex} writes 
   a~new file containing the code to typeset the glossary. \LaTeX\ then 
   reads this file on the next run. The \app{makeindex}
   application is automatically invoked by the helper
   \app{makeglossaries} script, which works out all the
   appropriate settings from the \file{.aux} file.

   \begin{enumerate}
    \item If you are using \sty{ngerman}\footnote{deprecated, use
\sty{babel} instead} or some other package that
makes the double-quote character \verb|"| a shorthand, then use
\cs{GlsSetQuote} to change this to some other character. For
example:
\begin{verbatim}
\GlsSetQuote{+}
\end{verbatim}
    Use this as soon as possible after you've loaded the
\styfmt{glossaries} package.

    \item Add \cs{makeglossaries} to your preamble (before you start
    defining your entries).

    \item Put
\begin{definition}
\cs{printglossary}\oarg{options}
\end{definition}
    where you want your list of entries (glossary) to appear. (The
    \code{sort} key isn't available in \meta{options}.)
    Alternatively, use
\begin{definition}
\cs{printglossaries}
\end{definition}
    which will display all glossaries (if you have more than one).
    This command doesn't have any arguments.

\begin{important}
All glossaries are sorted using the same method
which may be identified with one of the package options:
\pkgopt[standard]{sort} (default), \pkgopt[use]{sort}
or \pkgopt[def]{sort}.
\end{important}

    \item Run \LaTeX\ on your document. This creates files with the
    extensions \file{.glo} and \file{.ist} (for example, if your 
    \LaTeX\ document is called \file{myDoc.tex}, then you'll have 
    two extra files called \file{myDoc.glo} and \file{myDoc.ist}).
    If you look at your document at this point, you won't see the 
    glossary as it hasn't been created yet.

    \item Run \app{makeglossaries} with the base name of your 
    document (without the \file{.tex}) extension. If
    you have access to a terminal or a command prompt (for example, the
    MSDOS command prompt for Windows users or the bash console for
    Unix-like users) then you need to run the command:
\begin{verbatim}
makeglossaries myDoc
\end{verbatim}
   (Replace \file{myDoc} with the base name of your \LaTeX\
    document file without the \file{.tex} extension. 
    Avoid spaces in the file name.) If you don't have Perl installed
    use \app{makeglossaries-lite} instead:
\begin{verbatim}
makeglossaries-lite myDoc
\end{verbatim}

\begin{important}
Some beginners get confused by \app{makeglossaries} the
application (run as a system command) and \cs{makeglossaries}
the \LaTeX\ command which should be typed in the document
preamble. These are two different concepts that happen to have
similar looking names.
\end{important}

    If you don't know how to use the command prompt, then you can 
    probably configure your text editor to add 
    \app{makeglossaries} (or \app{makeglossaries-lite})
    as a build tool, but each editor has a
    different method of doing this, so I~can't give a~general
    description. You will have to check your editor's manual.
    If you still have problems, try adding the \pkgopt{automake}
    package option:
\begin{verbatim}
\usepackage[automake]{glossaries}
\end{verbatim}

    The default sort is word order (\qt{sea lion} comes before \qt{seal}). 
    If you want letter ordering you need to add the
    \pkgopt[letter]{order} package option
\begin{verbatim}
\usepackage[order=letter]{glossaries}
\end{verbatim}

    \item Once you have successfully completed the previous step,
    you can now run \LaTeX\ on your document again.
   \end{enumerate}
Here's a complete document (\file{myDoc.tex}):
\begin{verbatim}
\documentclass{article}

\usepackage{glossaries}

\makeglossaries % create makeindex files

\newglossaryentry{sample}{name={sample},
  description={an example}}

\begin{document}
A \gls{sample}.

\printglossaries % input files created by makeindex
\end{document}
\end{verbatim}
Document build:
\begin{verbatim}
pdflatex myDoc
makeglossaries myDoc
pdflatex myDoc
\end{verbatim}
or
\begin{verbatim}
pdflatex myDoc
makeglossaries-lite myDoc
pdflatex myDoc
\end{verbatim}


\item[]\optionlabel3:

   This option uses an application called
   \app{xindy} to sort the entries. This application is more
   flexible than \app{makeindex} and is able to sort extended
   Latin or non-Latin alphabets. It comes with both \TeX~Live and 
   MiK\TeX. Since \app{xindy} is a Perl script, you will also 
   need to ensure that Perl is installed. In a~similar way to \opt2, this 
   option involves making \LaTeX\ write the glossary information to 
   a~temporary file which \app{xindy} reads. Then \app{xindy} 
   writes a~new file containing the code to typeset the glossary. 
   \LaTeX\ then reads this file on the next run. The \app{xindy}
   application is automatically invoked by the helper
   \app{makeglossaries} script, which works out all the
   appropriate settings from the \file{.aux} file.

   \begin{enumerate}
     \item Add the \pkgopt{xindy} option to the \sty{glossaries}
package option list:
\begin{verbatim}
\usepackage[xindy]{glossaries}
\end{verbatim}

     \item Add \cs{makeglossaries} to your preamble (before you start
     defining your entries).

    \item Put
\begin{definition}
\cs{printglossary}\oarg{options}
\end{definition}
    where you want your list of entries (glossary) to appear. (The
    \code{sort} key isn't available in \meta{options}.)
    Alternatively, use
\begin{definition}
\cs{printglossaries}
\end{definition}

\begin{important}
All glossaries are sorted using the same method
which may be identified with one of the package options:
\pkgopt[standard]{sort} (default), \pkgopt[use]{sort}
or \pkgopt[def]{sort}.
\end{important}

    \item Run \LaTeX\ on your document. This creates files with the
    extensions \file{.glo} and \file{.xdy} (for example, if your 
    \LaTeX\ document is called \file{myDoc.tex}, then you'll have 
    two extra files called \file{myDoc.glo} and \file{myDoc.xdy}).
    If you look at your document at this point, you won't see the 
    glossary as it hasn't been created yet.

    \item Run \app{makeglossaries} with the base name of the
    document (omitting the \file{.tex} extension).
    If you have access to a terminal or a command prompt (for example, the
    MSDOS command prompt for Windows users or the bash console for
    Unix-like users) then you need to run the command:
\begin{verbatim}
makeglossaries myDoc
\end{verbatim}
    (Replace \file{myDoc} with the base name of your \LaTeX\
    document file without the \file{.tex} extension. 
     Avoid spaces in the file name. If you don't know
    how to use the command prompt, then as mentioned above, you may
    be able to configure your text editor to add
    \app{makeglossaries} as a build tool.

    The default sort is word order (\qt{sea lion} comes before
\qt{seal}). 
    If you want letter ordering you need to add the
    \code{order=letter} package option:
\begin{verbatim}
\usepackage[xindy,order=letter]{glossaries}
\end{verbatim}

    \item Once you have successfully completed the previous step,
    you can now run \LaTeX\ on your document again.

   \end{enumerate}
Here's a complete document (\file{myDoc.tex}):
\begin{verbatim}
\documentclass{article}

\usepackage[xindy]{glossaries}

\makeglossaries % create xindy files

\newglossaryentry{sample}{name={sample},
  description={an example}}

\begin{document}
A \gls{sample}.

\printglossaries % input files created by xindy
\end{document}
\end{verbatim}
Document build:
\begin{verbatim}
pdflatex myDoc
makeglossaries myDoc
pdflatex myDoc
\end{verbatim}

\item[]\optionlabel4:

   This requires the extension package \sty{glossaries-extra} and
   an application called \app{bib2gls}. This application is able
   to sort extended Latin or non-Latin alphabets. It comes with both
   \TeX~Live and MiK\TeX\ but requires at least Java~7. This method
   works differently to \optsand23. Instead of creating a file
   containing the code to typeset the glossary it creates a
   \file{.glstex} file containing the entry definitions fetched 
   from the \file{.bib} file (or files), but only those entries 
   that are required in the document are defined and they are 
   defined in the order obtained from the chosen sort method. 
   This means that you can just use
   \cs{printunsrtglossary} to display each glossary (or
   \cs{printunsrtglossaries} to display them all).

   \begin{enumerate}
   \item Add the \pkgopt{record} option to the \sty{glossaries-extra}
   package option list:
\begin{verbatim}
\usepackage[record]{glossaries-extra}
\end{verbatim}

   \item Add one or more 
\begin{definition}
\cs{GlsXtrLoadResources}[src=\marg{bib list},\meta{options}]
\end{definition}
   to your preamble where \meta{bib list} is the
   list of \file{.bib} files containing the entries. You may
   use different sort methods for each resource set. For example:
\begin{verbatim}
\usepackage[record,% using bib2gls
 abbreviations,
 symbols,
 numbers
]{glossaries-extra}

\GlsXtrLoadResources[
  src={terms},% entries in terms.bib
  type=main,% put these entries in the 'main' (default) list
  sort={de-CH-1996}% sort according to this locale
]
\GlsXtrLoadResources[
  src={abbrvs},% entries in abbrvs.bib
  type=abbreviations,% put these entries in the 'abbreviations' list
  sort={letter-case}% case-sensitive letter (non-locale) sort
]
\GlsXtrLoadResources[
  src={syms},% entries in syms.bib
  type=symbols,% put these entries in the 'symbols' list
  sort={use}% sort according to first use in the document
]
\GlsXtrLoadResources[
  src={constants},% entries in constants.bib
  type=numbers,% put these entries in the 'numbers' list
  sort-field={user1},% sort according to this field
  sort={double}% double-precision sort
]
\end{verbatim}
The last resource set assumes that the entries defined in the
file \file{constants.bib} have a number stored in the \code{user1}
field. For example:
\begin{verbatim}
@number{pi,
  name={\ensuremath{\pi}},
  description={pi},
  user1={3.141592654}
}
\end{verbatim}

   \item Put 
\begin{definition}
\cs{printunsrtglossary}[type=\marg{type},\meta{options}]
\end{definition}
    where you want your list of entries (glossary) to appear. (The
    \code{sort} key isn't available in \meta{options}. It needs to be
    used in \cs{GlsXtrLoadResources} instead.)
    Alternatively, use
\begin{definition}
\cs{printunsrtglossaries}
\end{definition}

    \item Run \LaTeX\ on your document. The \pkgopt{record} option 
    adds information to the \file{.aux} file that provides
    \app{bib2gls} with all required details for each resource set.
    For example, if the file is called \file{myDoc.tex}:
\begin{verbatim}
pdflatex myDoc
\end{verbatim}

    \item Run \app{bib2gls}
\begin{verbatim}
bib2gls myDoc
\end{verbatim}
or (if you need letter groups)
\begin{verbatim}
bib2gls --group myDoc
\end{verbatim}

    \item Run \LaTeX\ again.
   \end{enumerate}
Here's a complete document (\file{myDoc.tex}):
\begin{verbatim}
\documentclass{article}

\usepackage[record]{glossaries-extra}

\GlsXtrLoadResources % input file created by bib2gls
 [% instructions to bib2gls:
   src={entries}, % terms defined in entries.bib
   sort={en-GB}% sort according to this locale
 ]

\newglossaryentry{sample}{name={sample},
  description={an example}}

\begin{document}
A \gls{sample}.

\printunsrtglossaries % iterate over all defined entries
\end{document}
\end{verbatim}
The accompanying \file{entries.bib} file:
\begin{verbatim}
@entry{sample,
  name = {sample},
  description = {an example}
}
\end{verbatim}
Document build:
\begin{verbatim}
pdflatex myDoc
bib2gls myDoc
pdflatex myDoc
\end{verbatim}

\end{description}

If you are having difficulty integrating \app{makeglossaries}
into your document build process, you may want to consider using
\app{arara}, which is a Java application that searches the
document for special comment lines that tell \app{arara} which
applications to run. For example, the file \file{myDoc.tex} might
start with:
\begin{verbatim}
 % arara: pdflatex
 % arara: makeglossaries
 % arara: pdflatex
 \documentclass{article}
 \usepackage{glossaries}
 \makeglossaries
\end{verbatim}
then to build the document you just need the single system call:
\begin{verbatim}
 arara myDoc
\end{verbatim}
(The currently pending version 4.0 of \app{arara} has 
directives for \app{makeglossaries-lite} and \app{bib2gls}.
These aren't available in earlier versions, but you could try
copying and adapting the \file{makeglossaries.yaml} file and
replace \app{makeglossaries} as appropriate.)

When sorting the entries, the string comparisons are made according
to each entry's \code{sort} key. If this is omitted, the
\code{name} key is used. For example, recall the earlier
definition:
\begin{verbatim}
 \newglossaryentry{elite}
 {
   name={{\'e}lite},
   description={select group or class}
 }
\end{verbatim}
No \code{sort} key was used, so it's set to the same as the
\code{name} key: \verb|{\'e}lite|. How this is interpreted depends
on which option you have used:
\begin{description}
\item[\opt1:] By default, the accent command will be stripped so the
sort value will be \code{elite}. This will put the entry in the
\qt{E} letter group. However if you use the
\pkgopt[true]{sanitizesort} package option, the sort value will be
interpreted as the sequence of characters: \verb|{| \verb|\| \code{'} \code{e}
\verb|}| \code{l} \code{i} \code{t} and \code{e}. This will place
this entry in the \qt{symbols} group since it starts with a symbol.

\item[\opt2:] The sort key will be interpreted the sequence of characters:
\verb|{| \verb|\| \verb|'| \code{e} \verb|}| \code{l} \code{i} \code{t}
and \code{e}. The first character is an opening curly brace
\verb|{| so \app{makeindex} will put this entry in the
\qt{symbols} group. 

\item[\opt3:]
\app{xindy} disregards \LaTeX\ commands so it sorts on
\code{elite}, which puts this entry in the \qt{E} group.
If stripping all commands leads to an empty string (such as
\verb|\ensuremath{\emptyset}|) then \app{xindy} will fail, so in
these situations you need to provide an appropriate \code{sort}
value that \app{xindy} will accept.
\begin{important}
\app{xindy} merges entries with duplicate sort values. 
\app{xindy} forbids empty sort values.
A sort value may degrade into an empty or duplicate value once
\app{xindy} has stripped all commands and braces.
\end{important}

\item[\opt4:]
\app{bib2gls} has a primitive \LaTeX\ parser that recognises a
limited set of commands, which includes the standard accent commands
and some maths commands, so it can convert \verb|{\'e}lite| to
\code{\'elite}. It disregards unknown commands. This may lead to
an empty sort value, but \app{bib2gls} doesn't mind that.

Note that even if the name is given as \verb|{\'e}lite|, the letter
group heading (if the \verb|--group| switch is used) may end up with
the character \'E depending on the locale used by the sort
comparator. In this case you will need to ensure the document can
support this character either with \sty{inputenc} or by switching to
a \LaTeX\ engine with native UTF-8 support.

\end{description}


If the \sty{inputenc} package is used:
\begin{verbatim}
 \usepackage[utf8]{inputenc}
\end{verbatim}
and the entry is defined as:
\begin{alltt}
 \verb|\newglossaryentry{elite}|
 \verb|{|
   name=\verb|{{|\'e\verb|}lite},|
   description=\verb|{select group or class}|
 \verb|}|
\end{alltt}
then:
\begin{description}
\item[\opt1:] By default the sort value will be interpreted as
\code{elite} so the entry will be put in the \qt{E} letter group.
If you use the \pkgopt[true]{sanitizesort} package option, the
sort value will be interpreted as \code{\'elite} where \'e has
been sanitized (so it's no longer an active character and is in fact
seen as two octets 0xC3 0xA9) which will
put this entry before the \qt{A} letter group. (The group is
determined by the first octet 0xC3.)

\item[\opt2:] \app{makeindex} sees \code{\'e} as two octets
(0xC3 0xA9) rather than a single character so it tries to put
\code{\'elite} in the 0xC3 (\qt{\~A}) letter group (which, in this
case, comes after \qt{Z}).

\item[\opt3:] \app{xindy} will correctly recognise the sort value
\code{\'elite} and will place it in whatever letter group is
appropriate for the given language setting. (In English, this would
just be the \qt{E} letter group, but another language
might put it in the \qt{\'E} letter group.)

\item[\opt4:] The \sty{inputenc} package doesn't affect the
encoding used with \file{.bib} entry definitions, since these are dependent
on the encoding used to save the \file{.bib} file (although the
labels must still be ASCII). You can help \app{bib2gls} (and
JabRef) by putting an encoding comment at the start of the \file{.bib} file:
\begin{alltt}
\% Encoding: UTF-8
\end{alltt}

With the correct encoding set up, \app{bib2gls} will determine
that the sort value is \code{\'elite} and will place it in
whatever letter group is appropriate for the given sort rule.
For example, \verb|sort=en-GB| (or just \verb|sort=en|) will put 
\code{\'elite} in the \qt{E} letter group, but another language
might put it in the \qt{\'E} letter group.

\end{description}

Therefore if you have extended Latin or non-Latin characters, your
best option is to use either \app{xindy} (\opt3) or
\app{bib2gls} (\opt4) with the \sty{inputenc} or \sty{fontspec} package.
If you use \app{makeindex} (\opt2) you need to specify the
\code{sort} key like this:
\begin{verbatim}
\newglossaryentry{elite}
{
  name={{\'e}lite},
  sort={elite},
  description={select group or class}
}
\end{verbatim}
or
\begin{alltt}
\cs{newglossaryentry}\{elite\}
\{
  name=\{\{\'e\}lite\},
  sort=\{elite\},
  description=\{select group or class\}
\}
\end{alltt}
If you use \opt1, you may or may not need to use the \code{sort}
key, but you will need to be careful about fragile commands in the
\code{name} key if you don't set the \code{sort} key.

If you use \opt3 and the \code{name} only contains a command or
commands (such as \cs{P} or \verb|\ensuremath{\pi}|) you must add the
\code{sort} key. This is also advisable for the other options
(except \opt4), but is essential for \opt3. For example:
\begin{verbatim}
 \newglossaryentry{P}{name={\P},sort={P},
  description={paragraph symbol}}
\end{verbatim}

\section{Customising the Glossary}
\label{sec:glosstyle}

The default glossary style uses the \env{description} environment
to display the entry list. Each entry name is set in the optional
argument of \cs{item} which means that it will typically be
displayed in bold. You can switch to medium weight by redefining
\cs{glsnamefont}:
\begin{verbatim}
\renewcommand*{\glsnamefont}[1]{\textmd{#1}}
\end{verbatim}
Some classes and packages redefine the \env{description} environment
in such as way that's incompatible with the \styfmt{glossaries}
package. In which case you'll need to select a different glossary
style (see below).

By default, a~full stop is appended to the description (unless you
use \sty{glossaries-extra}). To prevent
this from happening use the \pkgopt{nopostdot} package option:
\begin{verbatim}
\usepackage[nopostdot]{glossaries}
\end{verbatim}
or to add it with \styfmt{glossaries-extra}:
\begin{verbatim}
\usepackage[postdot]{glossaries-extra}
\end{verbatim}

By default, a~location list is displayed for each entry (unless you
use \cs{printunsrtglossary} without \app{bib2gls}). This refers
to the document locations (for example, the page number) where the
entry has been referenced. If you use \optsor23 described in
\sectionref{sec:printglossaries} or \opt4 (with \app{bib2gls} and
\sty{glossaries-extra}) then location ranges will be compressed.
For example, if an entry was used on pages~1, 2 and~3, with
\optsor23 or \opt4 the location list will appear as 1--3, but with \opt1 it
will appear as 1, 2, 3. If you don't want the locations displayed
you can hide them using the \pkgopt{nonumberlist} package option:
\begin{verbatim}
\usepackage[nonumberlist]{glossaries}
\end{verbatim}
or with \app{bib2gls} use \code{save-locations=false} in the
optional argument of the appropriate \cs{GlsXtrLoadResources}
(it's possible to have some resource sets with locations and some
without).

Entries are grouped according to the first letter of
each entry's \code{sort} key. By default a~vertical gap is placed
between letter groups. You can suppress this with the
\pkgopt{nogroupskip} package option:
\begin{verbatim}
\usepackage[nogroupskip]{glossaries}
\end{verbatim}

If the default style doesn't suit your document, you can change the
style using:
\begin{definition}
\cs{setglossarystyle}\marg{style name}
\end{definition}
\href{http://www.dickimaw-books.com/gallery/glossaries-styles/}{There are a~number of predefined styles.} Glossaries can vary from
a~list of symbols with a~terse description to a~list of words or
phrases with descriptions that span multiple paragraphs, so there's
no \qt{one style fits all} solution. You need to choose a~style that
suits your document. For example:
\begin{verbatim}
\setglossarystyle{index}
\end{verbatim}
You can also use the \pkgopt{style} package option for the preloaded
styles. For example:
\begin{verbatim}
\usepackage[style=index]{glossaries}
\end{verbatim}

Examples:
\begin{enumerate}
 \item You have entries where the name is a~symbol and the
 description is a~brief phrase or short sentence. Try one of the 
 \qt{mcol} styles defined in the \sty{glossary-mcols} package. For example:
\begin{verbatim}
\usepackage[nopostdot]{glossaries}
\usepackage{glossary-mcols}
\setglossarystyle{mcolindex}
\end{verbatim}
or
\begin{verbatim}
\usepackage[stylemods={mcols},style=mcolindex]{glossaries-extra}
\end{verbatim}

 \item You have entries where the name is a~word or phrase and the
description spans multiple paragraphs. Try one of the \qt{altlist} styles. For
example:
\begin{verbatim}
\usepackage[nopostdot]{glossaries}
\setglossarystyle{altlist}
\end{verbatim}
or
\begin{verbatim}
\usepackage[stylemods,style=altlist]{glossaries-extra}
\end{verbatim}

 \item You have entries where the name is a~single word, the
  description is brief, and an associated symbol has been set.
  Use one of the styles that display the symbol (not all of them do). 
  For example, one of the tabular styles:
\begin{verbatim}
\usepackage[nopostdot,nonumberlist]{glossaries}
\setglossarystyle{long4col}
\end{verbatim}
or one of the \qt{tree} styles:
\begin{verbatim}
\usepackage[nopostdot,nonumberlist]{glossaries}
\setglossarystyle{tree}
\end{verbatim}
\end{enumerate}

If your glossary consists of a~list of abbreviations and you also want to
specify a~description as well as the long form, then you need to use
an abbreviation style that will suit the glossary style. For example,
use the \code{long-short-desc} acronym style:
\begin{verbatim}
\setacronymstyle{long-short-desc}
\end{verbatim}
Define the acronyms with a~description:
\begin{verbatim}
\newacronym
 [description={statistical pattern recognition technique}]
 {svm}{SVM}{support vector machine}
\end{verbatim}
Alternatively with \sty{glossaries-extra}:
\begin{verbatim}
\setabbreviationstyle{long-short-desc}

\newabbreviation
 [description={statistical pattern recognition technique}]
 {svm}{SVM}{support vector machine}
\end{verbatim}

Choose a~glossary style that suits wide entry names. For example:
\begin{verbatim}
\setglossarystyle{altlist}
\end{verbatim}

\section{Multiple Glossaries}
\label{sec:multigloss}

The \sty{glossaries} package predefines a~default \code{main}
glossary. When you define an entry (using one of the commands
described in \sectionref{sec:defterm}), that entry is automatically
assigned to the default glossary, unless you indicate otherwise 
using the \code{type} key. However you first need to
make sure the desired glossary has been defined. This is done using:
\begin{definition}
\cs{newglossary}\oarg{glg}\marg{label}\marg{gls}\marg{glo}\marg{title}
\end{definition}
The \meta{label} is a~label that uniquely identifies this new
glossary. As with other types of identifying labels, be careful not
to use active characters in \meta{label}. The final argument
\meta{title} is the section or chapter heading used by
\cs{printglossary} or \cs{printnoidxglossary}. The other arguments
indicate the file extensions used by
\app{makeindex}\slash\app{xindy} (described in
\sectionref{sec:printglossaries}). If you use \opt1 described
above (or \app{bib2gls} and \cs{printunsrtglossaries}), 
then the \meta{glg}, \meta{gls} and \meta{glo} arguments are
ignored, in which case you may prefer to use the starred version
where you don't specify the extensions:
\begin{definition}
\cs{newglossary*}\marg{label}\marg{title}
\end{definition}

In the case of \optsor23, all glossary definitions must come before
\cs{makeglossaries}. (Entries definitions should come after
\cs{makeglossaries}.) In the case of \opt4, all glossary definitions
must come before any \cs{GlsXtrLoadResources} that requires them.

Since it's quite common for documents to have both a~list of terms
and a~list of abbreviations, the \sty{glossaries} package provides the
package option \pkgopt{acronym} (or \pkgopt{acronyms}), which 
is a~convenient shortcut for
\begin{verbatim}
\newglossary[alg]{acronym}{acr}{acn}{\acronymname}
\end{verbatim}
The option also changes the behaviour of \cs{newacronym} so that acronyms
are automatically put in the list of acronyms instead of the main
glossary. The \sty{glossaries-extra} package also provides this
option for abbreviations defined using \cs{newacronym} but
additionally has the package option \pkgopt{abbreviations} to create
a list of abbreviations for \cs{newabbreviation}.

There are some other package options for creating commonly used
lists: \pkgopt{symbols} (lists of symbols), \pkgopt{numbers} (lists
of numbers), \pkgopt{index} (index of terms without descriptions defined with
\cs{newterm}\oarg{options}\marg{label}).

For example, suppose you want a~main glossary for terms, a~list of
acronyms and a~list of notation:
\begin{verbatim}
 \usepackage[acronyms]{glossaries}
 \newglossary[nlg]{notation}{not}{ntn}{Notation}
\end{verbatim}
After \cs{makeglossaries} (or \cs{makenoidxglossaries}) you can
define the entries in the preamble. For example:
\begin{verbatim}
 \newglossaryentry{gls:set}
 {% This entry goes in the `main' glossary
   name=set,
   description={A collection of distinct objects}
 }

 % This entry goes in the `acronym' glossary:
 \newacronym{svm}{svm}{support vector machine}

 \newglossaryentry{not:set}
 {% This entry goes in the `notation' glossary:
   type=notation,
   name={\ensuremath{\mathcal{S}}},
   description={A set},
   sort={S}}
\end{verbatim}
or if you don't like using \cs{ensuremath}:
\begin{verbatim}
 \newglossaryentry{not:set}
 {% This entry goes in the `notation' glossary:
   type=notation,
   name={$\mathcal{S}$},
   text={\mathcal{S}},
   description={A set},
   sort={S}}
\end{verbatim}

Each glossary is displayed using:
\begin{definition}
\cs{printnoidxglossary}[type=\meta{type}]
\end{definition}
(\opt1) or
\begin{definition}
\cs{printglossary}[type=\meta{type}]
\end{definition}
(\optsand23). Where \meta{type} is the glossary label. If the
type is omitted the default \code{main} glossary is assumed.

If you're using \app{bib2gls} then each glossary is displayed
using:
\begin{definition}
\cs{printunsrtglossary}[type=\meta{type}]
\end{definition}
With this method you don't use \cs{makeglossaries} or
\cs{makenoidxglossaries}. Instead you can assign the entry type
with the resource command. For example:
\begin{verbatim}
 \usepackage[record,abbreviations,symbols]{glossaries-extra}

 \GlsXtrLoadResources[
  src={terms}, % entries defined in terms.bib
  type={main}% put in main glossary
 ]
 \GlsXtrLoadResources[
  src={abbrvs}, % entries defined in abbrvs.bib
  type={abbreviations}% put in abbreviations glossary
 ]
 \GlsXtrLoadResources[
  src={syms}, % entries defined in syms.bib
  type={symbols}% put in symbols glossary
 ]
\end{verbatim}
Later in the document:
\begin{verbatim}
 \printunsrtglossary % main
 \printunsrtglossary[type=abbreviations]
 \printunsrtglossary[type=symbols]
\end{verbatim}

There's a~convenient shortcut that will display all the defined
glossaries depending on the indexing method:
\begin{definition}
\cs{printnoidxglossaries}
\end{definition}
(\opt1) or
\begin{definition}
\cs{printglossaries}
\end{definition}
(\optsand23) or (\sty{glossaries-extra} only)
\begin{definition}
\cs{printunsrtglossaries}
\end{definition}

If you use \opt1, you don't need to do anything else. If you use
\optsor23 with the \app{makeglossaries} Perl script or the
\app{makeglossaries-lite} Lua script, you
similarly don't need to do anything else. If you use \optsor23 
without the \app{makeglossaries} Perl script then you need to
make sure you run \app{makeindex}\slash\app{xindy} \emph{for
each defined glossary}. The \meta{gls} and \meta{glo} arguments of
\cs{newglossary} specify the file extensions to use instead of
\file{.gls} and \file{.glo}. The optional argument \meta{glg} is
the file extension for the transcript file. This should be different
for each glossary in case you need to check for
\app{makeindex}\slash\app{xindy} errors or warnings if things
go wrong.

For example, suppose you have three glossaries in your document
(\code{main}, \code{acronym} and \code{notation}), 
specified using:
\begin{verbatim}
\usepackage[acronyms]{glossaries}
\newglossary[nlg]{notation}{not}{ntn}{Notation}
\end{verbatim}
Then (assuming your \LaTeX\ document is in a~file called
\file{myDoc.tex}):

\begin{description}
\item\opt2
Either use one \app{makeglossaries} or
\app{makeglossaries-lite} call:
\begin{verbatim}
makeglossaries myDoc
\end{verbatim}
or
\begin{verbatim}
makeglossaries-lite myDoc
\end{verbatim}
Or you need to run \app{makeindex} three times:
\begin{verbatim}
makeindex -t myDoc.glg -s myDoc.ist -o myDoc.gls myDoc.glo
makeindex -t myDoc.alg -s myDoc.ist -o myDoc.acr myDoc.acn
makeindex -t myDoc.nlg -s myDoc.ist -o myDoc.not myDoc.ntn
\end{verbatim}

\item\opt3 
Either use one \app{makeglossaries} call:
\begin{verbatim}
makeglossaries myDoc
\end{verbatim}
Or you need to run \app{xindy} three times (be careful not to insert
line breaks where the line has wrapped.)
\begin{verbatim}
xindy  -L english -C utf8 -I xindy -M myDoc -t myDoc.glg
-o myDoc.gls myDoc.glo
xindy  -L english -C utf8 -I xindy -M myDoc -t myDoc.alg
-o myDoc.acr myDoc.acn
xindy  -L english -C utf8 -I xindy -M myDoc -t myDoc.nlg
-o myDoc.not myDoc.ntn
\end{verbatim}

\item\opt4
With \app{bib2gls} only one call is required:
\begin{verbatim}
pdflatex myDoc
bib2gls --group myDoc
pdflatex myDoc
\end{verbatim}
(Omit \verb|--group| if you don't need letter groups.)
\end{description}


\section[glossaries and hyperref]{\styfmt{glossaries} and \styfmt{hyperref}}
\label{sec:hyperref}

Take care if you use the \sty{glossaries} package with
\sty{hyperref}. Contrary to the usual advice that \sty{hyperref}
should be loaded last, \sty{glossaries} (and \sty{glossaries-extra}) 
must be loaded \emph{after} \sty{hyperref}:
\begin{verbatim}
\usepackage[colorlinks]{hyperref}
\usepackage{glossaries}
\end{verbatim}
If you use \sty{hyperref} make sure you use PDF\LaTeX\ rather than
the \LaTeX\ to DVI engine. The DVI format can't break hyperlinks across
a~line so long glossary entries (such as the full form of acronyms)
won't line wrap with the DVI engine. Also, hyperlinks in sub- or
superscripts aren't correctly sized with the DVI format.

By default, if the \sty{hyperref} package has been loaded, commands
like \cs{gls} will form a~hyperlink to the relevant entry in the
glossary. If you to disable this for \emph{all} your
glossaries, then use
\begin{definition}
\cs{glsdisablehyper}
\end{definition}
If you want hyperlinks suppressed for entries in specific
glossaries, then use the \pkgopt{nohypertypes} package option. For
example, if you don't want hyperlinks for entries in the \code{acronym} and
\code{notation} glossaries but you do want them for entries in the
\code{main} glossary, then do:
\begin{verbatim}
\usepackage[colorlinks]{hyperref}
\usepackage[acronym,nohypertypes={acronym,notation}]{glossaries}
\newglossary[nlg]{notation}{not}{ntn}{Notation}
\end{verbatim}

If you want the hyperlinks suppressed the first time an entry is
used, but you want hyperlinks for subsequence references then use
the \pkgopt[false]{hyperfirst} package option:
\begin{verbatim}
\usepackage[colorlinks]{hyperref}
\usepackage[hyperfirst=false]{glossaries}
\end{verbatim}

Take care not to use non-expandable commands in PDF bookmarks. This
isn't specific to the \sty{glossaries} package but is a~limitation
of PDF bookmarks. Non-expandable commands include commands like 
\cs{gls}, \cs{glspl}, \cs{Gls} and \cs{Glspl}. The \sty{hyperref}
package provides a~way of specifying alternative text for the PDF
bookmarks via \cs{texorpdfstring}. For example:
\begin{verbatim}
\section{The \texorpdfstring{\gls{fishage}}{Fish Age}}
\end{verbatim}
However, it's not a~good idea to use commands like \cs{gls} in
a~section heading as you'll end up with the table of contents in 
your location list. Instead you can use
\begin{definition}
\cs{glsentrytext}\marg{label}
\end{definition}
This is expandable provided that the \code{text} key doesn't
contain non-expandable code. For example, the following works:
\begin{verbatim}
\section{The \glsentrytext{fishage}}
\end{verbatim}
and it doesn't put the table of contents in the location list.

If you use \sty{glossaries-extra} then use the commands that are
provided specifically for use in section headers. For example:
\begin{verbatim}
\section{The \glsfmttext{fishage}}
\end{verbatim}

\section{Cross-References}
\label{sec:xr}

You can add a~reference to another entry in a~location list using
the \code{see=}\marg{label list} key when you define an entry. 
The referenced entry (or entries) must also be defined.

For example:
\begin{verbatim}
\longnewglossaryentry{devonian}{name={Devonian}}%
{%
  The geologic period spanning from the end of the 
  Silurian Period to the beginning of the Carboniferous Period.

  This age was known for its remarkable variety of 
  fish species.
}

\newglossaryentry{fishage}
{
  name={Fish Age},
  description={Common name for the Devonian period},
  see={devonian}
}
\end{verbatim}
The cross-reference will appear as \qt{\emph{see} Devonian}. You can
change the \qt{see} tag using the format
\code{see=}\oarg{tag}\meta{label}. For example:
\begin{verbatim}
\newglossaryentry{latinalph}
{
  name={Latin alphabet},
  description={alphabet consisting of the letters 
  a, \ldots, z, A, \ldots, Z},
  see=[see also]{exlatinalph}
}
\newglossaryentry{exlatinalph}
{
   name={extended Latin alphabet},
   description={The Latin alphabet extended to include 
   other letters such as ligatures or diacritics.}
}
\end{verbatim}
If you use the \code{see} key in the optional argument of
\cs{newacronym}, make sure you enclose the value in braces. For
example:
\begin{verbatim}
\newacronym{ksvm}{ksvm}{kernel support vector machine}
\newacronym
 [see={[see also]{ksvm}}]
 {svm}{svm}{support vector machine}
\end{verbatim}
The \sty{glossaries-extra} package provides a \code{seealso}
key. This doesn't allow a tag but behaves much like
\code{see=\{[\cs{seealsoname}]\marg{label}\}}. For example:
\begin{verbatim}
\newabbreviation{ksvm}{ksvm}{kernel support vector machine}
\newabbreviation
 [seealso={ksvm}]
 {svm}{svm}{support vector machine}
\end{verbatim}

Since the cross-reference appears in the location list, if you
suppress the location list using the \pkgopt{nonumberlist} package
option, then the cross-reference will also be suppressed. With
\app{bib2gls}, don't use the \pkgopt{nonumberlist} package
option. Instead use the \code{save-locations=false} in the resource
options. For example:
\begin{verbatim}
\usepackage[record,abbreviations,symbols]{glossaries-extra}

\GlsXtrLoadResources[
 src={terms}, % entries defined in terms.bib
 type={main}% put in main glossary
]
\GlsXtrLoadResources[
 src={abbrvs}, % entries defined in abbrvs.bib
 type={abbreviations},% put in abbreviations glossary
 save-locations=false% no number list for these entries
]
\GlsXtrLoadResources[
 src={syms}, % entries defined in syms.bib
 type={symbols}% put in symbols glossary
]
\end{verbatim}

\section{Further Information}
\label{sec:moreinfo}

\begin{itemize}
\item \href{http://mirrors.ctan.org/support/bib2gls/bib2gls-begin.pdf}{\styfmt{glossaries-extra}
and \appfmt{bib2gls}: an introductory guide}.
\item The main \sty{glossaries} \docref{user manual}{glossaries-user}.
\item The \href{http://www.dickimaw-books.com/faqs/glossariesfaq.html}{glossaries FAQ}.
\item \href{https://www.dickimaw-books.com/latex/buildglossaries/}{Incorporating
makeglossaries or makeglossaries-lite or bib2gls into the document
build}.
\item The \href{https://ctan.org/pkg/glossaries-extra}{\styfmt{glossaries-extra}} package.
\item The \href{https://ctan.org/pkg/bib2gls}{\appfmt{bib2gls}}
application.
\end{itemize}
\end{document}
