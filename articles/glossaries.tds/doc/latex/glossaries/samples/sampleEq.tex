 % This file is public domain
\documentclass[a4paper,12pt]{report}

\usepackage{amsmath}
\usepackage[colorlinks]{hyperref}
\usepackage[xindy,toc,ucmark,style=long3colheader,counter=equation]{glossaries}

\makeglossaries

\newcommand{\erf}{\operatorname{erf}}
\newcommand{\erfc}{\operatorname{erfc}}

\pagestyle{headings}

 % redefine the way hyperref creates the target for equations
 % so that the glossary links to equation numbers work

 \renewcommand*\theHequation{\theHchapter.\arabic{equation}}

 % Change the glossary headings

\renewcommand{\entryname}{Notation}
\renewcommand{\descriptionname}{Function Name}
\renewcommand{\pagelistname}{Number of Formula}

 % define glossary entries

\newglossaryentry{Gamma}{name=\ensuremath{\Gamma(z)},
description=Gamma function,
sort=Gamma}

\newglossaryentry{gamma}{name={\ensuremath{\gamma(\alpha,x)}},
description=Incomplete gamma function,
sort=gamma1}

\newglossaryentry{iGamma}{name={\ensuremath{\Gamma(\alpha,x)}},
description=Incomplete gamma function,
sort=Gamma2}

\newglossaryentry{psi}{name=\ensuremath{\psi(x)},
description=Psi function,sort=psi}

\newglossaryentry{erf}{name=\ensuremath{\erf(x)},
description=Error function,sort=erf}

\newglossaryentry{erfc}{name=\ensuremath{\erfc},
description=Complementary error function,sort=erfc}

\newglossaryentry{B}{name={\ensuremath{B(x,y)}},
description=Beta function,sort=B}

\newglossaryentry{Bx}{name={\ensuremath{B_x(p,q)}},
description=Incomplete beta function,sort=Bx}

\newglossaryentry{Tn}{name=\ensuremath{T_n(x)},
description=Chebyshev's polynomials of the first kind,sort=Tn}

\newglossaryentry{Un}{name=\ensuremath{U_n(x)},
description=Chebyshev's polynomials of the second kind,sort=Un}

\newglossaryentry{Hn}{name=\ensuremath{H_n(x)},
description=Hermite polynomials,sort=Hn}

\newglossaryentry{Ln}{name=\ensuremath{L_n^\alpha(x)},
description=Laguerre polynomials,sort=Lna}

\newglossaryentry{Znu}{name=\ensuremath{Z_\nu(z)},
description=Bessel functions,sort=Z}

\newglossaryentry{Phi}{name={\ensuremath{\Phi(\alpha,\gamma;z)}},
description=confluent hypergeometric function,sort=Pagz}

\newglossaryentry{knu}{name=\ensuremath{k_\nu(x)},
description=Bateman's function,sort=kv}

\newglossaryentry{Dp}{name=\ensuremath{D_p(z)},
description=Parabolic cylinder functions,sort=Dp}

\newglossaryentry{F}{name={\ensuremath{F(\phi,k)}},
description=Elliptical integral of the first kind,sort=Fpk}

\newglossaryentry{C}{name=\ensuremath{C},
description=Euler's constant,sort=C}

\newglossaryentry{G}{name=\ensuremath{G},
description=Catalan's constant,sort=G}

\begin{document}
\title{A Sample Document Using glossaries.sty}
\author{Nicola Talbot}
\maketitle

\begin{abstract}
This is a sample document illustrating the use of the \textsf{glossaries}
package.  The functions here have been taken from ``Tables of
Integrals, Series, and Products'' by I.S.~Gradshteyn and I.M~Ryzhik.
The glossary is a list of special functions, so
the equation number has been used rather than the page number.  This
can be done using the \texttt{counter=equation} package
option.
\end{abstract}

\printglossary[title={Index of Special Functions and Notations}]

\chapter{Gamma Functions}

\begin{equation}
\gls{Gamma} = \int_{0}^{\infty}e^{-t}t^{z-1}\,dt
\end{equation}

\verb|\ensuremath| is only required here if using
hyperlinks.
\begin{equation}
\glslink{Gamma}{\ensuremath{\Gamma(x+1)}} = x\Gamma(x)
\end{equation}

\begin{equation}
\gls{gamma} = \int_0^x e^{-t}t^{\alpha-1}\,dt
\end{equation}

\begin{equation}
\gls{iGamma} = \int_x^\infty e^{-t}t^{\alpha-1}\,dt
\end{equation}

\newpage

\begin{equation}
\gls{Gamma} = \Gamma(\alpha, x) + \gamma(\alpha, x)
\end{equation}

\begin{equation}
\gls{psi} = \frac{d}{dx}\ln\Gamma(x)
\end{equation}

\chapter{Error Functions}

\begin{equation}
\gls{erf} = \frac{2}{\surd\pi}\int_0^x e^{-t^2}\,dt
\end{equation}

\begin{equation}
\gls{erfc} = 1 - \erf(x)
\end{equation}

\chapter{Beta Function}

\begin{equation}
\gls{B} = 2\int_0^1 t^{x-1}(1-t^2)^{y-1}\,dt
\end{equation}
Alternatively:
\begin{equation}
\gls{B} = 2\int_0^{\frac\pi2}\sin^{2x-1}\phi\cos^{2y-1}\phi\,d\phi
\end{equation}

\begin{equation}
\gls{B} = \frac{\Gamma(x)\Gamma(y)}{\Gamma(x+y)} = B(y,x)
\end{equation}

\begin{equation}
\gls{Bx} = \int_0^x t^{p-1}(1-t)^{q-1}\,dt
\end{equation}

\chapter{Polynomials}

\section{Chebyshev's polynomials}

\begin{equation}
\gls{Tn} = \cos(n\arccos x)
\end{equation}

\begin{equation}
\gls{Un} = \frac{\sin[(n+1)\arccos x]}{\sin[\arccos x]}
\end{equation}

\section{Hermite polynomials}

\begin{equation}
\gls{Hn} = (-1)^n e^{x^2} \frac{d^n}{dx^n}(e^{-x^2})
\end{equation}

\section{Laguerre polynomials}

\begin{equation}
L_n^{\alpha} (x) = \frac{1}{n!}e^x x^{-\alpha}
\frac{d^n}{dx^n}(e^{-x}x^{n+\alpha})
\end{equation}

\chapter{Bessel Functions}

Bessel functions $Z_\nu$ are solutions of
\begin{equation}
\frac{d^2\glslink{Znu}{Z_\nu}}{dz^2}
+ \frac{1}{z}\,\frac{dZ_\nu}{dz} +
\left( 1-\frac{\nu^2}{z^2}Z_\nu = 0 \right)
\end{equation}

\chapter{Confluent hypergeometric function}

\begin{equation}
\gls{Phi} = 1 + \frac{\alpha}{\gamma}\,\frac{z}{1!}
+ \frac{\alpha(\alpha+1)}{\gamma(\gamma+1)}\,\frac{z^2}{2!}
+\frac{\alpha(\alpha+1)(\alpha+2)}{\gamma(\gamma+1)(\gamma+2)}\,
\frac{z^3}{3!} + \cdots
\end{equation}

\begin{equation}
\gls{knu} = \frac{2}{\pi}\int_0^{\pi/2}
\cos(x \tan\theta - \nu\theta)\,d\theta
\end{equation}

\chapter{Parabolic cylinder functions}

\begin{equation}
\gls{Dp} = 2^{\frac{p}{2}}e^{-\frac{z^2}{4}}
\left\{
\frac{\surd\pi}{\Gamma\left(\frac{1-p}{2}\right)}
\Phi\left(-\frac{p}{2},\frac{1}{2};\frac{z^2}{2}\right)
-\frac{\sqrt{2\pi}z}{\Gamma\left(-\frac{p}{2}\right)}
\Phi\left(\frac{1-p}{2},\frac{3}{2};\frac{z^2}{2}\right)
\right\}
\end{equation}

\chapter{Elliptical Integral of the First Kind}

\begin{equation}
\gls{F} = \int_0^\phi \frac{d\alpha}{\sqrt{1-k^2\sin^2\alpha}}
\end{equation}

\chapter{Constants}

\begin{equation}
\gls{C} = 0.577\,215\,664\,901\ldots
\end{equation}

\begin{equation}
\gls{G} = 0.915\,965\,594\ldots
\end{equation}

\end{document}
