\documentclass[11pt, oneside]{article}   	% use "amsart" instead of "article" for AMSLaTeX format
\usepackage{geometry}                		% See geometry.pdf to learn the layout options. There are lots.
\geometry{letterpaper}                   		% ... or a4paper or a5paper or ... 
%\geometry{landscape}                		% Activate for rotated page geometry
%\usepackage[parfill]{parskip}    		% Activate to begin paragraphs with an empty line rather than an indent
\usepackage{graphicx}				% Use pdf, png, jpg, or eps§ with pdflatex; use eps in DVI mode
								% TeX will automatically convert eps --> pdf in pdflatex		
\usepackage{amssymb}
\usepackage{color}
\newtheorem{definition}{Definition}
%SetFonts

%SetFonts

\usepackage{ragged2e}
\title{Time series analysis}
\author{Yapi Donatien Achou}
%\date{}							% Activate to display a given date or no date

\begin{document}
\maketitle
\section{What is a time series}
\textcolor{blue}{A time series is a set of observations $x_{t}$ recorded at a specific time $t$}. For discrete time series, the set $T$ at which the observations are recorded is discrete.
Example of time series: recording the arterial pressure every minute, recording the voltage of a computer every second.
%%%%%%%%%%%%%%%%%%%%%%%%%%%%%%%%%%%%%%%%%%%%%%%%%%%%%%%%%%%%%%%%%%%%%%%%%%%%%%%%%%
%%%%%%%%%%%%%%%%%%%%%%%%%%%%%%%%%%%%%%%%%%%%%%%%%%%%%%%%%%%%%%%%%%%%%%%%%%%%%%%%%%
\section{Objective and methodology of time series analysis}
\textcolor{blue}{The objective of time series analysis is to draw inferences from them}. What does it mean to draw inferences? Before answering that question let define the steps in time series analysis.
\begin{enumerate}
\item \textcolor{blue}{Select a family of probability models}
\item \textcolor{blue}{Estimate parameters. This will lead you to a specific model for the time series}
\item \textcolor{blue}{Check goodness of fit to the data}
\end{enumerate}
Now drawing inferences from the model means using the model \textcolor{blue}{to enhance our understanding of the underlying process that generated the time series}.
Or we can simply say without lost of generality that drawing inferences is an umbrella expression for different aspects that will eventually lead us to enhance our understanding of the underlying 
process that generated to time series. 
\justify
Drawing inferences as in 
\begin{itemize}
\item use the model to provide a compact description of the data
\item perform seasonal  adjustment
\item predict future values of a series (Predicting future sales for wine, predict future values of stock)
\item use time series model to estimate the probability of emptiness of a reservoir, in reservoir simulation
\item perform hypotheses testing
\end{itemize}
Now let get a bit technical
\section{Definition of a time series model}
\begin{definition}
\textcolor{blue}{A time series models} for the observed data $\{ x_{t}\}$ is \textcolor{blue}{a specification of the join distribution} (or possibly only the \textcolor{blue}{mean} and \textcolor{blue}{covariance}) of a sequence of random variables $\{ X_{t}\}$ of which $\{ x_{t}\}$ is postulated to be a realisation.
\end{definition}





\begin{thebibliography}{9}
\bibitem{petter} 
Petter J. Brocckwell, Richard A. Davis 
\textit{Introduction to time series and forecasting}. 
Springer Texts in Statistics, second edition.
 
\end{thebibliography}


\end{document}  