\documentclass[11pt, oneside]{article}   	% use "amsart" instead of "article" for AMSLaTeX format
\usepackage{geometry}                		% See geometry.pdf to learn the layout options. There are lots.
\geometry{letterpaper}                   		% ... or a4paper or a5paper or ... 
%\geometry{landscape}                		% Activate for rotated page geometry
%\usepackage[parfill]{parskip}    		% Activate to begin paragraphs with an empty line rather than an indent
\usepackage{graphicx}				% Use pdf, png, jpg, or eps§ with pdflatex; use eps in DVI mode
								% TeX will automatically convert eps --> pdf in pdflatex		
\usepackage{amssymb,amsmath}
\usepackage{amsthm}
\usepackage{float}

\usepackage{listings}
\usepackage[utf8]{inputenc}
\newcommand{\Var}{\operatorname{Var}}
\newcommand{\E}{\operatorname{E}}
\newcommand{\Cov}{\operatorname{Cov}}
% Default fixed font does not support bold face
\DeclareFixedFont{\ttb}{T1}{txtt}{bx}{n}{12} % for bold
\DeclareFixedFont{\ttm}{T1}{txtt}{m}{n}{12}  % for normal

\lstset{language=R,
    basicstyle=\small\ttfamily,
    stringstyle=\color{DarkGreen},
    otherkeywords={0,1,2,3,4,5,6,7,8,9},
    morekeywords={TRUE,FALSE},
    deletekeywords={data,frame,length,as,character},
    keywordstyle=\color{blue},
    commentstyle=\color{DarkGreen},
     %frame=single, % adds a frame around the code
     backgroundcolor=\color{lightgray},
}
\usepackage[svgnames]{xcolor}
\title{STAT211 Mandatory Homework 4}
\author{Yapi Donatien Achou}
%\date{}							% Activate to display a given date or no date

\begin{document}
\maketitle

\section{Problem 4.1}
Considere an ARMA(p,q) model 
\begin{equation}\label{eq:arma}
X_{t}-\sum_{k=1}^{p}\phi_{k}X_{t-k} = Z_{t}+\sum_{k=1}^{p}\theta_{k}Z_{t-k}
\end{equation}
\subsection{Part a: invertibility}
An ARMA(p,q) process $\{  X_{t}\}$ is invertible if there exist constant $\{  \pi_{j}\}$ such that 
\begin{equation}
\sum_{j=0}^{\infty}|\pi_{j}| \le \infty
\end{equation}
and 
\begin{equation}\label{eq:z}
Z_{t} = \sum_{j=0}^{\infty}\pi_{j}X_{t-j} \quad \text{for all t}.
\end{equation}
In other word $\{  X_{t}\}$ is invertible if $Z_{t}$ can be written as a linear combination of $X_{t-j}$, $j = 0,1, \dots, \infty$, \cite{petter}.

\begin{flushleft}
Invertibility is equivalent to 
\begin{equation}\label{eq:theta}
\theta(z) = 1+ \theta_{1}z+ \cdots + \theta_{q}z^{q}\neq 0 \quad \text{for all} \quad |z| \leq 1
\end{equation}
where $\theta(z)$ is the moving average polynomial.
\end{flushleft}

\begin{flushleft}
The process $X_{t}$ is invertible if and only if the zeros of the moving average polynomial $\theta(B)$ lie outside the unit circle.
%Now let prove that if a process is invertible then $Z_{t}$ is expressible in terms of $X_{s}$, $s\leq t$, given that $|\theta| < 1$.
\end{flushleft}

\subsection{Part b}
The sequence in (\ref{eq:z}) is determined by the relation
\begin{equation}\label{eq:main}
(1+\theta_{1}z +\theta_{2}z^{2} + \cdots + \theta_{q}z^{q})(\pi_{0} + \pi_{1}z + \cdots) = (1-\phi_{1}z-\phi_{2}z^{2}-\cdots-\phi_{p}z^{q}).
\end{equation}
Multiplying the left hand side together gives 
\begin{equation}
\begin{aligned}
(1+\theta_{1}z +\theta_{2}z^{2} + \cdots + \theta_{q}z^{q})(\pi_{0} + \pi_{1}z + \cdots)&=\pi_{0}+\pi_{1}z+\pi_{2}z^{2}+\cdots +\theta_{1}\pi_{0}z+\theta_{1}\pi_{1}z^{2}+\cdots+\theta_{2}\pi_{0}z^{2}\\
&=\pi_{0}+(\pi_{1}+\theta_{1}\pi_{0})z+(\pi_{2}+\theta_{1}\pi_{1}+\theta_{2}\pi_{0})z^{2}+ \cdots \nonumber
\end{aligned}
\end{equation}
and equation (\ref{eq:main}) can be rewritten as 
\begin{equation}\label{eq:main1}
\pi_{0}+(\pi_{1}+\theta_{1}\pi_{0})z+(\pi_{2}+\theta_{1}\pi_{1}+\theta_{2}\pi_{0})z^{2}+ \cdots = (1-\phi_{1}z-\phi_{2}z^{2}-\cdots-\phi_{p}z^{q}).
\end{equation}
And equating the coefficients of $z^{j}, j = 0, 1, \cdots$, we obtain
\begin{equation}
\begin{aligned}
\pi_{0} &= 1\\
\pi_{1}+\theta_{1}\pi_{0}&=-\phi_{1}\\
\pi_{2}+\theta_{1}\pi_{1}+\theta_{2}\pi_{0} &=-\phi_{2}\\
\vdots\nonumber
\end{aligned}
\end{equation}
or equivalently
\begin{equation}
\pi_{j}+\sum_{k=1}^{q}\theta_{k}\pi_{j-1} = -\phi_{j}, \quad j =0,1, \cdots
\end{equation}


%\begin{flushleft}
%\begin{proof}
%Equation (\ref{eq:arma}) can be written as
%\begin{equation}\label{eq:arma1}
%\phi(B)X_{t} = \theta(B)Z_{t}
%\end{equation}
%where $B$ is the backshift operator defined by 
%\begin{equation}
%B^{j}X_{t} = X_{t-j}
%\end{equation}
%\begin{equation}
%B^{j}Z_{t} = Z_{t-j}
%\end{equation}
% $\phi(.)$ and $\theta(.)$ are the p-th and q-th autoregressive and moving average polynomial 
% \begin{equation}\label{eq:ar}
% \phi(B) = 1-\phi_{1}B-\dots \phi_{p}B^{p}
% \end{equation}
%  \begin{equation}\label{eq:ma}
% \theta(B) = 1+\theta_{1}B+\dots \theta_{q}B^{q}.
% \end{equation}
% The n-th order Taylor series expansion of $\theta(B)$ around $a=0$ is given by
% \begin{equation}\label{eq:tayl}
% \theta(B) = \sum_{j=0}^{n}\frac{\theta^{n}(0)}{n!}B^{n}.
% \end{equation}
% Setting $\theta_{1} = \theta$ and $\theta(0) = 1$, the first order Taylor series expansion of $\theta(B)$ is given by 
% \begin{equation}
% \begin{aligned}
% \theta(B) &= \theta(0) + \theta^{\prime}(0)B\\
%               &= 1+\theta B
% \end{aligned}
% \end{equation}
% where $\theta^{\prime}(0)$ is the derivative of $\theta$ evaluated at 0.
% 
% \begin{flushleft}
% Now we know that the Maclaurin series for $\frac{1}{1-x}$ is 
% \begin{equation}\label{eq:mac}
% \frac{1}{1-x} = 1+x+x^{2}\cdots+ = \sum_{j=0}^{\infty}x^{j}, \quad \text{for $|x| < 1$}
% \end{equation}
% Setting 
% \begin{equation}
% 1-x = \theta(B) 
% \end{equation}
% we get 
% \begin{equation}
% x = 1-\theta(B) = 1- (1+\theta B) = -\theta B
% \end{equation}
% And inserting the value of $x$ back into (\ref{eq:mac}) we get
% \begin{equation}\label{eq:series}
% \frac{1}{\theta(B)} =  \sum_{j=0}^{\infty}(-\theta B)^{j} = \sum_{j=0}^{\infty}(-\theta)^{j} (B)^{j}, \quad \text{for $|\theta| <1$}
% \end{equation}
% Multiplying (\ref{eq:series}) by (\ref{eq:arma1}) give
% \begin{equation}\label{eq:arma2}
% \begin{aligned}
%Z_{t}&=\theta(B)\phi(B)X_{t} \\
%        &= \sum_{j=0}^{\infty}(-\theta)^{j} (B)^{j}\phi(B)X_{t}
%\end{aligned}
%\end{equation}
%From (\ref{eq:tayl}),the first order Taylor series expansion of $\phi(B)$ is given by 
%\begin{equation}
%\phi(B) = 1-\phi B
%\end{equation}
%So 
%\begin{equation}
%\begin{aligned}
%Z_{t} &= \sum_{j=0}^{\infty}(-\theta)^{j} (B)^{j}(1-\phi B)X_{t} \\
%        &=(1-\theta B+(-\theta)^{2} B^{2} + \cdots )(1-\phi B)X_{t}\\
%        &=(X_{t}-\phi BX_{t})(1-\theta B+(-\theta)^{2} B^{2} + \cdots )\\
%        &=(X_{t}-\phi X_{t-1})(1-\theta B+(-\theta)^{2} B^{2} + \cdots )\\
%        &=X_{t}-\theta BX_{t} + \theta^{2}B^{2}X_{t} - \theta^{3}B^{3}X_{t} + \cdots -\phi X_{t-1} + \phi\theta BX_{t-1} -\phi\theta^{2}B^{2} X_{t-1} + \phi\theta^{3}B^{3} X_{t-1}-\cdots\\
%        &=X_{t}-\theta X_{t-1} + \theta^{2}X_{t-2} - \theta^{3}X_{t-3} + \cdots -\phi X_{t-1} + \phi\theta X_{t-2} -\phi\theta^{2} X_{t-3} + \phi\theta^{3} X_{t-4}-\cdots\\
%        &=X_{t}-(\theta + \phi)X_{t-1} + (\theta + \phi)\theta X_{t-2} -(\theta + \phi)\theta^{2} X_{t-3} + (\theta + \phi)\theta^{3} X_{t-4} - \cdots \\
%        &=X_{t} -(\theta + \phi)(X_{t-1} -\theta X_{t-2} + \theta^{2} X_{t-3} - \theta^{3} X_{t-4} + \cdots)\\
%        &=X_{t}-(\theta + \phi)\sum_{j=1}^{\infty}(-\theta)^{j-1}X_{t-j}
%\end{aligned}
%\end{equation}
%So that $Z_{t}$ can be expressed in terms of the present and past values of the process $X_{s}$, $s\leq t$
% \end{flushleft}
% 
% 
% 
%\end{proof}
%\end{flushleft}





\section{All R code Code}

\begin{lstlisting}
\end{lstlisting}



\begin{thebibliography}{9}
\bibitem{petter} 
Petter J. Brockwell. Richard A. Davis
\textit{Introduction to Time Series and Forecasting}. 
Springer. Second edition. 2001
 
\end{thebibliography}


\end{document}  