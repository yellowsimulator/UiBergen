\documentclass[11pt, oneside]{article}   	% use "amsart" instead of "article" for AMSLaTeX format
\usepackage{geometry}                		% See geometry.pdf to learn the layout options. There are lots.
\geometry{letterpaper}                   		% ... or a4paper or a5paper or ... 
%\geometry{landscape}                		% Activate for rotated page geometry
%\usepackage[parfill]{parskip}    		% Activate to begin paragraphs with an empty line rather than an indent
\usepackage{graphicx}				% Use pdf, png, jpg, or eps§ with pdflatex; use eps in DVI mode
								% TeX will automatically convert eps --> pdf in pdflatex		
\usepackage{amssymb,amsmath}
\usepackage{amsthm}
\usepackage{float}
\usepackage{bm}


\usepackage{listings}
\usepackage[utf8]{inputenc}
\newcommand{\Var}{\operatorname{Var}}
\newcommand{\E}{\operatorname{E}}
\newcommand{\Cov}{\operatorname{Cov}}
% Default fixed font does not support bold face
\DeclareFixedFont{\ttb}{T1}{txtt}{bx}{n}{12} % for bold
\DeclareFixedFont{\ttm}{T1}{txtt}{m}{n}{12}  % for normal

\lstset{language=R,
    basicstyle=\small\ttfamily,
    stringstyle=\color{DarkGreen},
    otherkeywords={0,1,2,3,4,5,6,7,8,9},
    morekeywords={TRUE,FALSE},
    deletekeywords={data,frame,length,as,character},
    keywordstyle=\color{blue},
    commentstyle=\color{DarkGreen},
     %frame=single, % adds a frame around the code
     backgroundcolor=\color{lightgray},
}
\usepackage[svgnames]{xcolor}
\title{Applied statistics Homework 2}
\author{Yapi Donatien Achou}
%\date{}							% Activate to display a given date or no date

\begin{document}

\maketitle
\tableofcontents
\newpage
 
\section{Problem 2.1}
We will learn about the following new R functions, as well as new functions
\begin{lstlisting}
read.csv()
t.test()
shapiro.test()
read.table()
\end{lstlisting}
\subsection{Part1: t-test}
When the test statistic follows the normal distribution (gaussian distribution), \textcolor{blue}{a t-test} is commonly applied to test the mean of a normally distributed population. This is achieved in R by using the command \textcolor{blue}{t.test()}. For that you need a sample $X$, the mean $\mu$ of the population from which the sample $X$ is drown, the mean $\mu_{x}$
of the sample, and the significance level $\alpha$. Assuming a significance level $\alpha=5\%$, the test is given by
\begin{lstlisting}[mathescape=true]
t.test(X-$\mu_{X}$, alternative=two.side, conf.level=0.95)
\end{lstlisting}
More option are 
\begin{lstlisting}[mathescape=true]
t.test(x,y=NULL,
alternative=c(two.sided,less, greater ),
mu=0, paired=FALSE, var.equal=FALSE,
conf.level=0.95,
formula, data,subset,na.action...
)
\end{lstlisting}
If the pvalue is greater than the significance level $\alpha$, we can conclude that the null hypothesis is plausible

\subsection{Part 2: Hypotheses test formulation}
We would like to test the hypotheses that the average yield of barley is greater than 150
\begin{equation}
\begin{split}
H_{0}: \mu & =150 \\
H_{1}:\mu &> 150
\end{split}
\end{equation}
\clearpage
\begin{lstlisting}[mathescape=false]
Barley <- read.csv("Barley.csv")

mean <- mean(Barley$barley)
alpha <- 0.1 
level <- 1-alpha
test <- t.test(Barley$barley, mu=150, alternative="greater", conf.level=level)
>>
>>
	One Sample t-test

data:  Barley$barley
t = 1.3607, df = 399, p-value = 0.08719
alternative hypothesis: true mean is greater than 150
90 percent confidence interval:
 150.1199      Inf
sample estimates:
mean of x 
 152.1175 
\end{lstlisting}
The one sided 90$\%$ confidence interval suggests that the mean barley yield is likely to be greater than 150.1199.
%The $p-value$ of $0.08719$ tells us that if the mean barley yield was 150, the probability of selecting a sample with mean yield greater than or equal to 150 is 
%$8.719$.
The pvalue of $0.08719$ is less than the significance level $\alpha = 0.1$, we therefor reject the null hypothesis that the mean is equal to 150 in favour of the alternative hypothesis

Now if the significance level $\alpha=0.05$  then the pvalue of $0.08719$ is greater than  the the the the significance level, than we would not reject $H_{0}$ in favour of $H_{1}$.
\subsection{Part 3}


\subsection{Note}
Reject $H_{0}$ if pvalue is less than significance level, and not reject $H_{0}$ otherwise. The significance level $\alpha$ is the probability of rejecting the null hypothesis when it is true.
The pvale, is the probability of obtaining a result at least as extreme, given that the null hypothesis is true. The result is \textcolor{blue}{statistically significant}, by the standard of the study when 
\begin{equation}
pvalue < \alpha
\end{equation}



\begin{thebibliography}{9}
\bibitem{petter} 
Petter J. Brockwell. Richard A. Davis
\textit{Introduction to Time Series and Forecasting}. 
Springer. Second edition. 2001
 
\end{thebibliography}


\end{document}  